\section{HISS kriterierna} % (fold)
\label{sec:hiss}
    För en bedömning av rapporten delas bedömningen upp genom att nyttja de så kallade HISS kriterierna vid Chalmers, Helhetsintryck, Innehåll och förståelse, Struktur och Språk.

    \subsection{Helhetsintryck} % (fold)
    \label{sub:helhetsintryck}
        Upplevelsen av rapporten som helhet är något svag. Många delar saknar en tydlig struktur, språket är under all kritik och flertalet bilder är av undermålig kvalité. Av vad som framgår av rapporten är resultatet av arbetet en välgrundad produkt, men rapporten i sig själv är inte av en hög kvalité.
    % subsection helhetsintryck (end)

    \subsection{Innehåll och förståelse} % (fold)
    \label{sub:innehall}
        Innehållet i rapporten är god, även om det presenteras på ett undermålig vis. Rapporten redogör i detalj för hur produkten har producerats och vilka verktyg de har använt för att producera resultatet.
    % subsection innehall (end)

    \subsection{Struktur} % (fold)
    \label{sub:sturktur}
        Den övergripande dispositionen och strukturen av rapporten är god, den följer en standardiserad mall och det är enkelt att få en överblick kring vilka kapitel som finns och vad dessa innehåller. \bigskip

        Däremot är strukturen inom varje enskilt kapitel inte tydligt, det finns ingen så kallad ''röd tråd'' för läsaren att följa. Detta gäller för kapitlen 1-4, medan kapitel 5 ''Resultat'' har en mycket tydlig struktur och leder läsaren framåt. Kapitel 6 är även den tydlig, dock saknar vi som opponenter delar av kapitlet så en utförlig kommentar kring hur strukturen fungerar är inte möjlig att producera.
    % subsection sturktur (end)

    \subsection{Språk} % (fold)
    \label{sub:spr_k}
        Språket i rapporten är ofta subjektiv, något som inte passar i en teknisk rapport. Flera delar av texten upplevs som direkt tagna från olika marknadsföringsmaterial, andra delar är inte grammatiska korrekta.
    % subsection spr_k (end)
    
% section formalia (end)