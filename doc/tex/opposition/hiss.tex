\section{HISS-kriterierna} % (fold)
\label{sec:hiss}
    Bedömningen av rapporten delas upp genom att nyttja de så kallade HISS-kriterierna vid Chalmers, Helhetsintryck, Innehåll och förståelse, Struktur och Språk.

    \subsection{Helhetsintryck} % (fold)
    \label{sub:helhetsintryck}
    \paragraph{Syfte och struktur}
        Upplevelsen av rapporten som helhet är något svag. Många delar saknar en tydlig struktur, språket är under all kritik och flertalet bilder är av undermålig kvalité. Av vad som framgår av rapporten är resultatet av arbetet en välgrundad produkt, men rapporten i sig själv är inte av en hög kvalité.

    \paragraph{Ämnesbeskrivning}

    \paragraph{Formalia}

    % subsection helhetsintryck (end)

    \subsection{Innehåll och förståelse} % (fold)
    \label{sub:innehall}
    Innehållet i rapporten är god, även om det presenteras på ett undermålig vis. Rapporten redogör i detalj för hur produkten har producerats och vilka verktyg de har använt för att producera resultatet.
    \paragraph{Relationen syfte – resultatanalys – diskussion (rödtråd)}

    \paragraph{Teori- och metodval}

    \paragraph{Argumentation}

    \paragraph{Resultatredovisning}

    \paragraph{Analys i förhållande till teori/metod  }

    \paragraph{Kritiskt förhållningssätt till resultat}

    \paragraph{Faktainsamling}

    % subsection innehall (end)

    \subsection{Struktur} % (fold)
    \label{sub:struktur}
    \paragraph{Kapitel och avsnitt}
        Den övergripande dispositionen och strukturen av rapporten är god, den följer en standardiserad mall och det är enkelt att få en överblick kring vilka kapitel som finns och vad dessa innehåller. \bigskip
     
    \paragraph{Styckehantering}

    \paragraph{Referenshantering}
         Rapporten innehåller flera referenser, men placeringen av referenserna är inte inte konsekvent. Rapporten verkar använda IEEE standarden, vilket innebär att referensnumret ska placeras innanför skiljetecknet med ett mellanslag från det föregående ordet.
    \paragraph{Tabeller och figurer}

    
        Däremot är strukturen inom varje enskilt kapitel inte tydligt, det finns ingen så kallad ''röd tråd'' för läsaren att följa. Detta gäller för kapitlen 1-4, medan kapitel 5 ''Resultat'' har en mycket tydlig struktur och leder läsaren framåt. Kapitel 6 är även den tydlig, dock saknar vi som opponenter delar av kapitlet så en utförlig kommentar kring hur strukturen fungerar är inte möjlig att producera.
    
    \paragraph{Kapitel och avsnitt}

    % subsection struktur (end)

    \subsection{Språk} % (fold)
    \label{sub:sprak}
    Rapporten är i sin helhet skriven på engelska och det kan antas att det ej är författarnas modersmål. Med detta i åtanke är språket överlag acceptabelt. Det är samtidigt märkbart att ordval och grammatik ibland har en till synes svensk prägel.

    \paragraph{Meningsbyggnad}
        Övergripande är meningsbyggnaden acceptabel och förståelig. Det förekommer några meningar där innebörden är otydlig och att man som läsare får stanna upp och se till stycket som helhet för att förstå vad som menas. Exempel på detta är första stycket i avsnitt \emph{4.1 Back End} och andra styckets andra mening i avsnitt \emph{4.2.3 Design}. Det förekommer också flera grammatiska felaktigheter.

    \paragraph{Ordval och begrepp}
        Merparten av ordvalen uppfattas som korrekta för de tekniska begrepp och områden som beskrivs. Det finns dock en del avvikelser, där det mest frekventa är en återkommande användning av förstapersonsperspektiv (eng. \emph{we}). 

    \paragraph{Textens stil}
        Språket i rapporten är bitvis subjektivt, vilket ej är lämpligt i en teknisk rapport. Det förekommer att delar av texten upplevs som tagna eller inspirerade från marknadsföringsmaterial, exempelvis i avsnitten \emph{1.2 The Contractor: Volvo Group Telematics} och \emph{3.5 WebGL}. Överlag uppfattas textens stil som något ojämn och varierande i formalitet. Det förekommer uttryck som ''all of a sudden'' och ''it just so happens'' och sista stycket i avsnitt \emph{6.4.2.1 UX (User Experience)} bör också nämnas.

    \paragraph{Korrektur}
        En konsekvens av att rapporten inte är helt färdigställd är att en gedigen korrekturläsning antagligen inte har genomförts. Rapporten innehåller en del stavfel men påverkar inte förståelsen innehållet. Av något mindre karaktär kan nämnas att vissa förekommande ord ej bör ha versal begynnelsebokstav, såsom \emph{iOS} och beståndsdelarna i \emph{MVC/MVVM}.

    % subsection sprak (end)

% section formalia (end)