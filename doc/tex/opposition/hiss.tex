\section{HISS-kriterierna} % (fold)
\label{sec:hiss}
    Bedömningen av rapporten delas upp genom att nyttja de så kallade HISS-kriterierna vid Chalmers; \textbf{H}elhetsintryck, \textbf{I}nnehåll och förståelse, \textbf{S}truktur och \textbf{S}pråk.

    \subsection{Helhetsintryck} % (fold)
    \label{sub:helhetsintryck}
        Upplevelsen av rapporten som helhet är något svag. Många delar saknar en tydlig struktur, språket är under all kritik och flertalet bilder är av undermålig kvalitet. Av vad som framgår av rapporten är resultatet av arbetet en välgrundad produkt, men rapporten i sig själv är inte väl utformad.

        \paragraph{Syfte och struktur}
            Syftet och den övergripande strukturen är tydligt utformat, genom en översikt av de inledande sidorna i rapporten kan läsaren ta till sig det avsedda syftet, samt det övergripande upplägget som rapporten strävar att jobba efter.

        \paragraph{Ämnesbeskrivning}
            Genom rapportens syfte och titel leds läsaren till att förvänta sig en genomgång av 3D-simulering, dock ligger tyngden i rapporten på utvecklingen av strukturen för en webbapplikation. Förhållandet mellan hur mycket som skrivs om implementationen av de olika delarna som krävs för en webbapplikation och simuleringsdelen i rapporten är väldigt ojämn. Resultatet tar dock upp frågeställningen och besvarar den på ett strukturerat sätt.

        \paragraph{Formalia}
            Det finns brister gällande formalia i rapporten. Författarna har valt att inkludera tre stycken ikoner för att indikera olika saker, något som är vanligt i läroböcker. Detta är inget som bör förekomma i en teknisk rapport och deras ''tips'' innebär för de mesta helt ovidkommande information. Denna information är något som bör strykas helt från rapporten. Antingen är texten relevant och då ska den inkluderas i brödtexten, eller så ska den strykas från rapporten helt. Den enda ikon som kan anses vara försvarbar är den som indikerar att terminal instruktioner, men dessa instruktioner kan visas på ett lämpligare vis, så som ett annat typsnitt på den texten.
    % subsection helhetsintryck (end)

    \subsection{Innehåll och förståelse} % (fold)
    \label{sub:innehall}
    Innehållet i rapporten är god, även om det presenteras på ett undermålig vis. Rapporten redogör i detalj för hur produkten har producerats och vilka verktyg de har använt för att producera resultatet. Författarna har dock inte visat på en förståelse av framställande av rapporten, då det saknas återkoppling från resultat- och diskussionskapitlet till implementationskapitlet.

    \paragraph{Relationen syfte – resultatanalys – diskussion (rödtråd)}
        Syftet i rapporten är mer skrivet som en kravspecifikation än ett övergripande syfte och utifrån detta syfte är sedan frågeställningen härledd. Avgränsningen är att anses som bra, även om den första punkten i avsnittet 1.4 inte är en relevant avgränsning för rapporten. Resultatkapitlet besvarar frågeställningen och ett eget avsnitt som heter just krav från uppdragsgivaren, men det refereras inte direkt till syftet. Dock är syftet implicit besvarat, då de har utformat den produkt med de verktyg som uppdragsgivaren efterfrågar och produktens egenskaper redogörs för i resultatdelen. I diskussionskapitlet hänvisar författarna tillbaka till inledningen, men inte till syftet, istället är det snarare deras kravspecifikationen som tas upp i sammanfattningen.

    \paragraph{Teori- och metodval}
        Författarna beskriver i metodkapitlet en struktur för hur applikationen förväntas arbeta, vilka verktyg som kommer att användas samt en uppdelning av ansvar mellan de båda författarna. Det redogörs inte för någon litteratur, eller annan källa, som stöd för tillvägagångssättet. Det nämns inte heller någon formell utvecklingsmetod för arbetet. I det utkast som vi opponenter har mottagit förkommer ingen kritisk diskussion, avsnittet 6.2 finns endast som en rubrik så det finns för närvarande ingen text som förhåller sig till vald metod.

    \paragraph{Argumentation}
        Flertalet påståenden görs utan något starkt underlag, till exempel andra och tredje meningen i första stycket i 1.1 och sista meningen i andra stycket 1.1, men flera ställen förekommer. På andra ställen görs uttalanden och antaganden utifrån svaga grunder, som till exempel första meningen i 3.3.1. I detta exempel utgår författarna från att en viss typ av implementation är ledande inom området, med stöd av en kurva över antalet jobbannonser. Det ska dock påtalas att det finns påståenden i rapportens som är uppbackade av riktiga referenser.

    \paragraph{Resultatredovisning}
        Resultaten redovisas på ett klart och tydligt sätt, med en tydlig struktur till hjälp. Författarna redovisar både vilka krav de har och inte har uppfyllt.

    \paragraph{Analys i förhållande till teori/metod  }
        Då diskussionsdelen inte är färdigställd i denna rapport kan det tänkas att en analys gällande metoden kommer att ta plats där, men för närvarande finns ingen analys gällande metod presenterad i rapporten.

    \paragraph{Kritiskt förhållningssätt till resultat}
        På samma grund som föregående stycke saknas ett kritiskt förhållningssätt till resultat, dock finns det ett påbörjat avsnitt gällande vidareutveckling, avsnitt 6.4. I detta avsnitt redogörs för hur en vidare utveckling av databasen kan utformas vilket kan anses som viss kritik mot sitt resultat, att den nuvarande databasen är begränsad.

    \paragraph{Faktainsamling}
        Rapporten innehåller 38 stycken källor var av merparten är att anses som goda källor. Det finns däremot antaganden i rapporten som bygger på osäkra källor så som referens nummer 1, 5 och 7. Referens 1 och 5 är bloggar utan någon tydlig ansvarig utgivare eller författare vilket gör att dessa sidor inte kan anses som trovärdiga källor i ett examensarbete. Referens 7 är en andrahandskälla och författaren till den texten är inte en expert på området enligt hans beskrivning, utan framstår snarare som en marknadsföringsperson. För att använda den här typen av referenser bör en kritisk diskussion kring källorna hållas för att föra fram det budskap som rapportens författare är ute efter.

    % subsection innehall (end)

    \subsection{Struktur} % (fold)
    \label{sub:struktur}
    \paragraph{Kapitel och avsnitt}
        Den övergripande dispositionen och strukturen av rapporten är god, den följer en standardiserad mall och det är enkelt att få en överblick kring vilka kapitel som finns och vad dessa innehåller. \bigskip
     
    \paragraph{Styckehantering}

    \paragraph{Referenshantering}
         Rapporten innehåller flera referenser, men placeringen av referenserna är inte inte konsekvent. Rapporten verkar använda IEEE standarden, vilket innebär att referensnumret ska placeras innanför skiljetecknet med ett mellanslag från det föregående ordet.
    
    \paragraph{Tabeller och figurer}

    
        Däremot är strukturen inom varje enskilt kapitel inte tydligt, det finns ingen så kallad ''röd tråd'' för läsaren att följa. Detta gäller för kapitlen 1-4, medan kapitel 5 ''Resultat'' har en mycket tydlig struktur och leder läsaren framåt. Kapitel 6 är även den tydlig, dock saknar vi som opponenter delar av kapitlet så en utförlig kommentar kring hur strukturen fungerar är inte möjlig att producera.

    % subsection struktur (end)

    \subsection{Språk} % (fold)
    \label{sub:sprak}
    Rapporten är i sin helhet skriven på engelska och det kan antas att det ej är författarnas modersmål. Med detta i åtanke är språket överlag acceptabelt. Det är samtidigt märkbart att ordval och grammatik ibland har en till synes svensk prägel.

    \paragraph{Meningsbyggnad}
        Övergripande är meningsbyggnaden acceptabel och förståelig. Det förekommer några meningar där innebörden är otydlig och att man som läsare får stanna upp och se till stycket som helhet för att förstå vad som menas. Exempel på detta är första stycket i avsnitt 4.1 och andra styckets andra mening i avsnitt 4.2.3. Det förekommer också flera grammatiska felaktigheter.

    \paragraph{Ordval och begrepp}
        Merparten av ordvalen uppfattas som korrekta för de tekniska begrepp och områden som beskrivs. Det finns dock en del avvikelser, där det mest frekventa är en återkommande användning av förstapersonsperspektiv (eng. \emph{we}). 

    \paragraph{Textens stil}
        Språket i rapporten är bitvis subjektivt, vilket ej är lämpligt i en teknisk rapport. Det förekommer att delar av texten upplevs som tagna eller inspirerade från marknadsföringsmaterial, exempelvis i avsnitten 1.2 och 3.5. Överlag uppfattas textens stil som något ojämn och varierande i formalitet. Det förekommer uttryck som ''all of a sudden'' och ''it just so happens'' och sista stycket i avsnitt 6.4.2.1 bör också nämnas.

    \paragraph{Korrektur}
        En konsekvens av att rapporten inte är helt färdigställd är att en gedigen korrekturläsning antagligen inte har genomförts. Rapporten innehåller en del stavfel men påverkar inte förståelsen innehållet. Av något mindre karaktär kan nämnas att vissa förekommande ord ej bör ha versal begynnelsebokstav, såsom \emph{iOS} och beståndsdelarna i \emph{MVC/MVVM}.

    % subsection sprak (end)

% section formalia (end)