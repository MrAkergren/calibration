\section{HISS-kriterierna} % (fold)
\label{sec:hiss}
    Bedömningen av rapporten delas upp genom att nyttja de så kallade HISS-kriterierna vid Chalmers; \textbf{H}elhetsintryck, \textbf{I}nnehåll och förståelse, \textbf{S}truktur och \textbf{S}pråk.

    \subsection{Helhetsintryck} % (fold)
    \label{sub:helhetsintryck}
        Rapporten upplevs som helhet något svag. Många delar saknar tydlig struktur, språket är under all kritik och flertalet bilder är av undermålig kvalitet. Av det som framgår i rapporten verkar resultatet av arbetet vara en välgrundad produkt, men rapporten är i sig själv inte väl utformad.

        \paragraph{Syfte och struktur}
            Syftet och den övergripande strukturen är tydligt utformade. Genom en översikt av de inledande sidorna i rapporten kan läsaren skapa sig en uppfattning om syftet, samt det övergripande upplägg som rapporten strävar efter.

        \paragraph{Ämnesbeskrivning}
            Genom rapportens syfte och titel leds läsaren till att förvänta sig en genomgång av 3D-simulering, dock ligger rapportens tyngd på utvecklingen av en webbapplikations struktur. Förhållandet mellan hur mycket som skrivits om simuleringsdelen och implementationen av de olika delarna som krävs för en webb\-applikation är väldigt ojämn i rapporten. Resultatet tar dock upp frågeställningen och besvarar den på ett strukturerat sätt.

        \paragraph{Formalia}
            Det finns brister gällande formalia i rapporten. Författarna har valt att inkludera tre stycken ikoner för att indikera olika saker, något som är vanligt i läro- och instruktionsböcker. Detta är inget som bör förekomma i en teknisk rapport och deras ''tips'' innebär för de mesta helt ovidkommande information. Antingen är texten relevant, och ska då inkluderas i brödtexten, eller så ska den helt strykas från rapporten. Den enda ikon som skulle kunna anses vara försvarbar är den som indikerar terminalinstruktioner, men dessa instruktioner bör visas på ett mer lämpligt vis, så som ett annat typsnitt på den aktuella texten.
    % subsection helhetsintryck (end)

    \subsection{Innehåll och förståelse} % (fold)
    \label{sub:innehall}
    Innehållet i rapporten är god, även om det presenteras på ett undermålig vis. Rapport\-en redogör i detalj för hur produkten har producerats och vilka verktyg som har använt för att producera resultatet. Författarna har dock inte visat på en förståelse av framställande av rapporten, då det saknas återkoppling från resultat- och diskussionskapitlet till implementationskapitlet.

    \paragraph{Relationen syfte – resultatanalys – diskussion (rödtråd)}
        Syftet i rapporten är mer skrivet som en kravspecifikation än ett övergripande syfte och utifrån detta syfte är sedan frågeställningen härledd. Avgränsningen är att anses som bra, även om den första punkten i avsnittet 1.4 inte är en relevant avgränsning för rapporten. Resultatkapitlet besvarar frågeställningen och ett eget avsnitt som heter ''krav från uppdragsgivaren'', men det refereras inte direkt till syftet. Dock är syftet implicit besvarat, då de har utformat den produkt med de verktyg som uppdragsgivaren efterfrågar och produktens egenskaper redogörs för i resultatdelen. I diskussionskapitlet hänvisar författarna tillbaka till inledningen, men inte till syftet, istället är det snarare deras kravspecifikationen som tas upp i sammanfattningen.

    \paragraph{Teori- och metodval}
        Författarna beskriver i metodkapitlet en struktur för hur applikationen förväntas arbeta, vilka verktyg som kommer att användas samt en uppdelning av ansvar mellan de båda författarna. Det redogörs inte för någon litteratur, eller annan källa, som stöd för tillvägagångssättet. Det nämns inte heller någon formell utvecklingsmetod för arbetet. I det utkast som vi opponenter har mottagit förkommer ingen kritisk diskussion, avsnittet 6.2 finns endast som en rubrik så det finns för närvarande ingen text som förhåller sig till vald metod.

    \paragraph{Argumentation}
        Flertalet påståenden görs utan något starkt underlag, till exempel andra och tredje meningen i första stycket i 1.1 och sista meningen i andra stycket 1.1, men flera ställen förekommer. På andra ställen görs uttalanden och antaganden utifrån svaga grunder, som till exempel första meningen i 3.3.1. I detta exempel utgår författarna från att en viss typ av implementation är ledande inom området, med stöd av en kurva över antalet jobbannonser. Det ska dock påtalas att det finns påståenden i rapportens som är uppbackade av riktiga referenser.

    \paragraph{Resultatredovisning}
        Resultaten redovisas på ett klart och tydligt sätt, med en tydlig struktur till hjälp. Författarna redovisar både vilka krav de har och inte har uppfyllt.

    \paragraph{Analys i förhållande till teori/metod  }
        Då diskussionsdelen inte är färdigställd i denna rapport kan det tänkas att en analys gällande metoden kommer att ta plats där, men för närvarande finns ingen analys gällande metod presenterad i rapporten.

    \paragraph{Kritiskt förhållningssätt till resultat}
        På samma grund som föregående stycke saknas ett kritiskt förhållningssätt till resultat, dock finns det ett påbörjat avsnitt gällande vidareutveckling, avsnitt 6.4. I detta avsnitt redogörs för hur en vidare utveckling av databasen kan utformas vilket kan anses som viss kritik mot sitt resultat, att den nuvarande databasen är begränsad.

    \paragraph{Faktainsamling}
        Rapporten innehåller 38 stycken källor var av merparten är att anses som goda källor. Det finns däremot antaganden i rapporten som bygger på osäkra källor så som referens nummer 1, 5 och 7. Referens 1 och 5 är bloggar utan någon tydlig ansvarig utgivare eller författare vilket gör att dessa sidor inte kan anses som trovärdiga källor i ett examensarbete. Referens 7 är en andrahandskälla och författaren till den texten är inte en expert på området enligt hans beskrivning, utan framstår snarare som en marknadsföringsperson. För att använda den här typen av referenser bör en kritisk diskussion kring källorna hållas för att föra fram det budskap som rapportens författare är ute efter.

    % subsection innehall (end)

    \subsection{Struktur} % (fold)
    \label{sub:struktur}
    \paragraph{Kapitel och avsnitt}
        Den övergripande dispositionen och strukturen av rapporten är god. En standardiserad mall följs och det är enkelt att få en överblick över kapitel och deras innehåll. När man tittar närmre på uppdelning av avsnitt så kan det upplevas som att alltför många korta avsnitt på fjärde nivån i hierarkin, framför allt i kapitel 4. Strukturen inom enskilda kapitel är däremot inte lika tydlig, det är svårt för läsaren att finna en ''röd tråd'' att följa. Detta gäller kapitlen 1-4, medan kapitel 5 istället har en mycket tydlig struktur som leder läsaren framåt. Även kapitel 6 är tydligt men vi som opponenter saknar delar av kapitlet, så en utförlig kommentar kring hur strukturen fungerar kan ej ges.
     
    \paragraph{Styckehantering}
        I huvudsak är fungerar styckeindelningen väl och ger en begriplig uppdelning av innehållet. Vissa avsnitt innehåller omotiverade uppdelningar av stycken, såsom 3.4.4 och delar av kapitel 4, medan avsnitten i kapitlen 5 och 6 är mer balanserade. På sidorna 14 och 26 påträffas märkligt stora utrymmen mellan stycken.

    \paragraph{Referenshantering}
        Rapporten innehåller flera referenser men placeringen är inte konsekvent. Den standard som verkar användas är IEEE, vilket innebär att referensnumret ska placeras innanför skiljetecknet med ett mellanslag från det föregående ordet. Vidare saknas referenser till externt inhämtade figurer, vilka är åtminstone figurerna 1.1 och 3.1. I avsnitt 1.2 refereras till \emph{Volvo Group Telematics} men ingen korrekt referens är angiven. Referenser till två olika länkar hos \emph{caniuse.com} förekommer i löpande text i avsnitt 3.1.1 och 3.5.2 men är inte inkluderade i referenssamlingen. I avsnitt 3.2.2, på sidan 12, förekommer en referens som ges som fotnot istället för att hänvisa till referenssamlingen. Det verkar också som att en del av appendix saknas då flera referenser hänvisar till \emph{Appendix ??}.
    
    \paragraph{Tabeller och figurer}
        Figurer och tabeller har utsatta rubriker som är lätta att se. Knappt hälften av figurerna har rubriker som är tydligt beskrivande medan resterande är svårbegripliga vid anblick av figurförteckningen. Placeringen av figurer är varierande. Vissa figurer fungerar utmärkt i text och sammanhang, såsom 4.1, medan andra är sämre placerade. I flera fall omnämns figurer inte över huvud taget i texten, vilket är fallet med exempelvis 3.2, 3.3 och 4.6. Antalet figurer i rapporten som helhet är i överkant och detta är som mest tydligt i kapitel 4, där avsikten att förklara och tydliggöra programmerad kod snarare får motsatt effekt.
    

    % subsection struktur (end)

    \subsection{Språk} % (fold)
    \label{sub:sprak}
    Rapporten är i sin helhet skriven på engelska och det kan antas att det ej är författarnas modersmål. Med detta i åtanke är språket överlag acceptabelt. Det är samtidigt märkbart att ordval och grammatik ibland har en till synes svensk prägel. Under \emph{Terminology}, där begrepp och förkortningar förklaras, skulle samtliga \emph{Short for} kunna tas bort då det är självklart i sammanhanget att de är akronymer.

    \paragraph{Meningsbyggnad}
        Övergripande är meningsbyggnaden acceptabel och förståelig. Det förekommer några meningar där innebörden är otydlig och att man som läsare får stanna upp och se till stycket som helhet för att förstå vad som menas. Exempel på detta är första stycket i avsnitt 4.1 och andra styckets andra mening i avsnitt 4.2.3. Det förekommer också flera grammatiska felaktigheter.

    \paragraph{Ordval och begrepp}
        Merparten av ordvalen uppfattas som korrekta för de tekniska begrepp och områden som beskrivs. Det finns dock avvikelser, där det mest frekventa är en återkommande användning av förstapersonsperspektiv (eng. \emph{we}). 

    \paragraph{Textens stil}
        Språket i rapporten är bitvis subjektivt, vilket ej är lämpligt i en teknisk rapport. Delar av texten har stora likheter med text som kan hittas i marknadsföringsmaterial, exempelvis avsnitten 1.2 och 3.5. Överlag uppfattas textens stil som något ojämn och varierande i formalitet. Det förekommer uttryck som ''all of a sudden'' och ''it just so happens'' och sista stycket i avsnitt 6.4.2.1 bör också nämnas. Antalet listor, både numrerade och i punktform, skulle kunna minskas.

    \paragraph{Korrektur}
        En konsekvens av att rapporten inte är helt färdigställd är att en gedigen korrekturläsning antagligen inte har genomförts. Rapporten innehåller en del stavfel men påverkar inte förståelsen innehållet. Ytterligare ett resultat av att en korrekturläsning inte verkar ha genomförts är den inkonsekventa behandlingen av förkortningar, vilket blir ett störningsmoment vid läsning. På vissa ställen i texten skrivs det ''Ett Ord (EO)'' medan på andra ställen ''EO (Ett Ord)''. Av något mindre karaktär kan nämnas att vissa förekommande ord ej bör ha versal begynnelsebokstav, såsom \emph{iOS} och beståndsdelarna i \emph{MVC/MVVM}. 

    % subsection sprak (end)
