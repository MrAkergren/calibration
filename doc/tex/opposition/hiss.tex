\section{HISS kriterierna} % (fold)
\label{sec:hiss}
    För en bedömning av rapporten delas bedömningen upp genom att nyttja de så kallade HISS kriterierna vid Chalmers; \textbf{H}elhetsintryck, \textbf{I}nnehåll och förståelse, \textbf{S}truktur och \textbf{S}pråk.

    \subsection{Helhetsintryck} % (fold)
    \label{sub:helhetsintryck}
        Upplevelsen av rapporten som helhet är något svag. Många delar saknar en tydlig struktur, språket är under all kritik och flertalet bilder är av undermålig kvalitet. Av vad som framgår av rapporten är resultatet av arbetet en välgrundad produkt, men rapporten i sig själv är inte väl utformad.

        \paragraph{Syfte och struktur}
            Syftet och den övergripande strukturen är tydligt utformat, genom en översikt av de inledande sidorna i rapporten kan läsaren ta till sig det avsedda syftet, samt det övergripande upplägget som rapporten strävar att jobba efter.

        \paragraph{Ämnesbeskrivning}
            Genom rapportens syfte och titel leds läsaren till att förvänta sig en genomgång av 3D simulering, dock ligger tyngden i rapporten på utvecklingen av strukturen för en webbapplikation. Förhållandet mellan hur mycket som skrivs om implementationen av de olika delarna som krävs för en webbapplikation och simuleringsdelen i rapporten är väldigt ojämn. Resultatet tar dock upp frågeställningen och besvarar den på ett strukturerat sätt.

        \paragraph{Formalia}
            Det finns brister gällande formalia i rapporten. Författarna har valt att inkludera tre stycken ikoner för att indikera olika saker, något som är vanligt i läroböcker. Detta är inget som bör förekomma i en teknisk rapport och deras ''tips'' innebär för de mesta helt ovidkommande information. Denna information är något som bör strykas helt från rapporten. Antingen är texten relevant och då ska den inkluderas i brödtexten, eller så ska den strykas från rapporten helt. Den enda ikon som kan anses vara försvarbar är den som indikerar att terminal instruktioner, men dessa instruktioner kan visas på ett lämpligare vis, så som ett annat typsnitt på den texten.
    % subsection helhetsintryck (end)

    \subsection{Innehåll och förståelse} % (fold)
    \label{sub:innehall}
    Innehållet i rapporten är god, även om det presenteras på ett undermålig vis. Rapporten redogör i detalj för hur produkten har producerats och vilka verktyg de har använt för att producera resultatet.

    \paragraph{Relationen syfte – resultatanalys – diskussion (rödtråd)}
        Syftet i rapporten är mer skriver som en krav specifikation än ett övergripande syfte och utifrån detta syfte är sedan frågeställningen härledd. Avgränsningen är att anses som bra, även om den första punkten i avsnittet 1.4 inte är en relevant avgränsning för rapporten. Resultatkapitlet besvarar frågeställningen och ett eget avsnitt som heter just krav från uppdragsgivaren, men det refereras inte direkt till syftet. Dock är syftet implicit besvarat, då de har utformat den produkt med de verktyg som uppdragsgivaren efterfrågar och produktens egenskaper redogörs för i resultatdelen. I diskussionskapitlet hänvisar författarna tillbaka till inledningen, men inte till syftet, istället är det snarare deras kravspecifikationen som tas upp i sammanfattningen.

    \paragraph{Teori- och metodval}

    \paragraph{Argumentation}

    \paragraph{Resultatredovisning}

    \paragraph{Analys i förhållande till teori/metod  }

    \paragraph{Kritiskt förhållningssätt till resultat}

    \paragraph{Faktainsamling}

    % subsection innehall (end)

    \subsection{Struktur} % (fold)
    \label{sub:sturktur}
    \paragraph{Kapitel och avsnitt}
        Den övergripande dispositionen och strukturen av rapporten är god, den följer en standardiserad mall och det är enkelt att få en överblick kring vilka kapitel som finns och vad dessa innehåller. \bigskip
     
    \paragraph{Styckehantering}

    \paragraph{Referenshantering}
         Rapporten innehåller flera referenser, men placeringen av referenserna är inte inte konsekvent. Rapporten verkar använda IEEE standarden, vilket innebär att referensnumret ska placeras innanför skiljetecknet med ett mellanslag från det föregående ordet.
    
    \paragraph{Tabeller och figurer}

    
        Däremot är strukturen inom varje enskilt kapitel inte tydligt, det finns ingen så kallad ''röd tråd'' för läsaren att följa. Detta gäller för kapitlen 1-4, medan kapitel 5 ''Resultat'' har en mycket tydlig struktur och leder läsaren framåt. Kapitel 6 är även den tydlig, dock saknar vi som opponenter delar av kapitlet så en utförlig kommentar kring hur strukturen fungerar är inte möjlig att producera.

    % subsection sturktur (end)

    \subsection{Språk} % (fold)
    \label{sub:sprak}

    \paragraph{Meningsbyggnad}
        Språket i rapporten är ofta subjektiv, något som inte passar i en teknisk rapport. Flera delar av texten upplevs som direkt tagna från olika marknadsföringsmaterial, andra delar är inte grammatiska korrekta.

    \paragraph{Ordval och begrepp}

    \paragraph{Textens stil}

    \paragraph{Korrektur}

    % subsection sprak (end)

% section formalia (end)