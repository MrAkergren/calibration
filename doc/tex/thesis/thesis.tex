\documentclass[a4paper, 11pt]{article}

%%--LANGUAGE AND ENCODING--%%
\usepackage[swedish]{babel}
\usepackage[english,cleanlook]{isodate}%
\usepackage[utf8]{inputenc}
\usepackage[T1]{fontenc}
\usepackage{csquotes}

\usepackage[yyyymmdd]{datetime}
\renewcommand{\dateseparator}{--}

\usepackage{xfrac}
\usepackage{lscape}
%%--BIBLOPGRAPHY--%%
\usepackage[backend=biber, natbib=true, urldate=iso8601, maxnames=2, minnames=1, maxbibnames=10, minbibnames=6, citestyle=numeric-comp, sorting=none, firstinits=true]{biblatex}

%%--SPACING AND MARGIN--%
\usepackage[textwidth=140mm]{geometry}
%\usepackage[margin=3.5cm, top=2.5cm]{geometry}
\setlength{\parindent}{0mm}


%%--SANS-SERIF FONTS FOR SECTIONS--%%
\usepackage{sectsty}
\usepackage{helvet}
\allsectionsfont{\bfseries\sffamily}

%%Links within the doc%%
\usepackage[hidelinks]{hyperref}

%%--GRAPHICS--%%  (Requires preamble)
\usepackage{floatrow}
\newfloatcommand{capbtabbox}{table}[][\FBwidth]

\usepackage{tikz}
\usetikzlibrary{backgrounds,calc,positioning,shapes,arrows}


% % Declaration of different types of figures for use in flow charts
\tikzstyle{startstop} = [rectangle, rounded corners, minimum width=3cm, minimum height=1cm,text centered, draw=black, fill=red!30]
\tikzstyle{io} = [trapezium, trapezium left angle=70, trapezium right angle=110, minimum width=2cm, minimum height=1cm, text centered, draw=black, fill=blue!30]
\tikzstyle{process} = [rectangle, minimum width=3cm, minimum height=1cm, text centered, draw=black, fill=orange!30]
\tikzstyle{decision} = [diamond, minimum width=3cm, minimum height=1cm, text centered, draw=black, fill=green!30]
\tikzstyle{arrow} = [draw, thick,-latex',>=stealth]


% For use in method figure
\tikzset{
    vertex/.style = {
        circle,
        fill            = black,
        outer sep = 2pt,
        inner sep = 1pt,
    }
}
\tikzset{My Arrow Style/.style={single arrow, fill=blue!50, anchor=base, align=center,text width=11mm,text height=1mm}}
\definecolor{light-gray}{gray}{0.95}

% For use in flow charts
\tikzstyle{startstop} = [rectangle, rounded corners, minimum width=3cm, minimum height=1cm,text centered, draw=black, fill=red!30]
\tikzstyle{input} = [trapezium, trapezium left angle=70, trapezium right angle=110, minimum width=2cm, minimum height=1cm, text centered, draw=black, fill=blue!30]
\tikzstyle{action} = [rectangle, minimum width=3cm, minimum height=1cm, text centered, draw=black, fill=orange!30]
\tikzstyle{decision} = [diamond, minimum width=3cm, minimum height=1cm, text centered, draw=black, fill=green!30]
\tikzstyle{arrow} = [draw, thick,-latex',>=stealth]

\usepackage{colortbl}
\usepackage{pgfplots}
\usepackage{pgfplotstable}
\pgfplotsset{compat=1.8}
\pgfplotstableset{
    every head row/.style={ 
        output empty row,
    },
    /color cells/min/.initial=0,
    /color cells/max/.initial=1000,
    /color cells/textcolor/.initial=,
    %
    % Usage: 'color cells={min=<value which is mapped to lowest color>, 
    %   max = <value which is mapped to largest>}
    color cells/.code={%
        \pgfqkeys{/color cells}{#1}%
        \pgfkeysalso{%
            postproc cell content/.code={%
                %
                \begingroup
                %
                % acquire the value before any number printer changed
                % it:
                \pgfkeysgetvalue{/pgfplots/table/@preprocessed cell content}\value
                \ifx\value\empty
                    \endgroup
                \else
                \pgfmathfloatparsenumber{\value}%
                \pgfmathfloattofixed{\pgfmathresult}%
                \let\value=\pgfmathresult
                %
                % map that value:
                \pgfplotscolormapaccess
                    [\pgfkeysvalueof{/color cells/min}:\pgfkeysvalueof{/color cells/max}]
                    {\value}
                    {\pgfkeysvalueof{/pgfplots/colormap name}}%
                % now, \pgfmathresult contains {<R>,<G>,<B>}
                % 
                % acquire the value AFTER any preprocessor or
                % typesetter (like number printer) worked on it:
                \pgfkeysgetvalue{/pgfplots/table/@cell content}\typesetvalue
                \pgfkeysgetvalue{/color cells/textcolor}\textcolorvalue
                %
                % tex-expansion control
                % see http://tex.stackexchange.com/questions/12668/where-do-i-start-latex-programming/27589#27589
                \toks0=\expandafter{\typesetvalue}%
                \xdef\temp{%
                    \noexpand\pgfkeysalso{%
                        @cell content={%
                            \noexpand\cellcolor[rgb]{\pgfmathresult}%
                            \noexpand\definecolor{mapped color}{rgb}{\pgfmathresult}%
                            \ifx\textcolorvalue\empty
                            \else
                                \noexpand\color{\textcolorvalue}%
                            \fi
                            \the\toks0 %
                        }%
                    }%
                }%
                \endgroup
                \temp
                \fi
            }%
        }%
    }
}

\usepackage{graphicx}
\usepackage{caption}
\usepackage[justification=centering]{caption}
\usepackage{subcaption}
\usepackage{threeparttable}
\usepackage{mathrsfs}
\makeatletter 
\g@addto@macro\TPT@defaults{\footnotesize} 
\makeatother
%%--ADVANCE TABULARS--%%
\usepackage{tabularx}
\usepackage{booktabs}
\def\arraystretch{1.3}

%PREAMBLE%
%%-SECTION NUMMBERING DEPTH-%%
%\setcounter{secnumdepth}{3} %3=Default

\hyphenation{över-förings-hastig-heten pa-nel-en instruktion-er sol-panel-en fatt-ades anse nu-varande av-sedda fungera-nde  kommunikations-stand-ard-er version-er mot-svarande enkorts-dator enhet-ens exempel-vis operativ-system doku-ment-ation platt-forms-oberoende system-utvecklings-metoden} 

%%-GRAPHICS-%%
\DeclareGraphicsExtensions{.pdf,.png,.jpg}

%%-BIBLIOGRAPHY-%%
%Adds references library and formats it.
% To  refere to a reference in the library use  \cite{} for ieee
\addbibresource{ref.bib} \setlength{\bibitemsep}{\baselineskip} 
%Always shows the authors in bibliography as Lastname, Firstname
%\DeclareNameAlias{sortname}{last-first} 

%%-DOCUMENT INFORMATION-%%
%Header/Footer%
\author{Svedberg, Pär\\ \texttt{svpar@student.chalmers.se}  \\ 
            19821112--7652 \and
            Åkergren, Oskar\\ \texttt{akergren@student.chalmers.se}  \\ 19880508--7114
}
\title{\underline{UTKAST} \\ Kalibrering av ljussensor \\ för Parans solpanel \vspace{1cm}}

\date{\vspace{8cm}\today}

\begin{document}
\maketitle
\begin{center}
    Version 0.17    
\end{center}

\thispagestyle{empty}

\newpage
\setcounter{page}{1}
\pagenumbering{roman}

\renewcommand{\abstractname}{Sammandrag}
\begin{abstract}
    Detta projekt syftar till att möjliggöra en automatisk kalibrering av fotosensorn på en solpanel som aktivt följer solen för att maximera dess solintag som nyttjas till belysning, en kalibrering som tidigare utförts manuellt. Projektet har genomförts i iterationer med stöd av en 'design science research'-metod och är fokuserat på två huvudsakliga områden, dels en effektiv kalibreringsalgoritm och dels en kommunikationslösning mellan solpanelen och det rum som panelen lyser upp. Projektet har resulterat i en färdig mjukvarulösning som kalibrerar panelen och projektet har lämnat två förslag på lösningar gällande kommunikationen, där vi rekommenderar att använda panelens egna fiberkablar som datamedium. 
\end{abstract}

\renewcommand{\abstractname}{Abstract}
\begin{abstract}
    The aim of this project is to automatically calibrate a photo sensor on a sun panel that is following the sun in its path, to maximize the light intake that is to be used as a light source in doors, a calibration that was originally made by hand. The project was performed in iterations according to a design science research method and has two main focus tasks. The first is to develop a calibration algorithm and the second is suggest a communication solution between the room and the panel. The result of the project is a fully functioning calibration application, and two suggestions for the communication where the fibre optic cables is used to transport the data.
\end{abstract}

\newpage
\subsection*{Förord} % (fold)
\label{sub:f_rord}
    Detta examensarbete skrivs inom ramen för kursen LMTX38, Examensarbete vid Data- och informationsteknik vårterminen 2015 på Chalmers tekniska högskola. \bigskip

    Författarna vill rikta ett stort tack till Parans Solar Lightning AB för tillhandahållande av materiel som projektet nyttjat sig av, lokal att arbeta i och stöd vid arbetet. Ett särskilt tack riktas till handledare Karl Nilsson vid Parans för förslaget till examensarbetets problemområde och avsatt tid för handledning av projektet. Vi riktar även ett tack till Simon Larsson vid Parans för bollande av idéer och stöttande vid utvecklingsarbetet. Vidare vill vi tacka handledare Lennart Hansson vid Chalmers för hans stöd vid skrivande av denna rapport och förslag för att driva arbetet framåt. Ett avslutande tack skickas till Sakib Sistek vid Chalmers för hans förmedling av kontakten till Parans och hans hjälp vid förberedningen till detta examensarbete.

% subsection f_rord (end)

\newpage

\subsection*{Beteckningar} % (fold)
\label{sub:beteckningar}
    \begin{tabularx}{\textwidth}{@{}rX}
        C & Programmeringsspråk \\
        I/O & Input/Output \\
        I²C & Standard för synkron seriell datakommunikation \\
        lm & lumen, SI-enhet för ljusflöde \\
        lux & SI-enhet för belysning, 1 lux = 1 $\sfrac{\text{lm}}{\text{m}^2}$ \\
        RS-232 & Standard för asynkron seriell datakommunikation \\
        UART & Universal Asynchronous Receiver/Transmitter, gränssnitt för seriell kommunikation
        
    \end{tabularx}
% subsection beteckningar (end)

\newpage
\tableofcontents
\listoffigures
\listoftables

\newpage

\setcounter{page}{1}
\pagenumbering{arabic}

\section{Introduktion} % (fold)
\label{sec:indroduktion}

    \subsection{Bakgrund} % (fold)
    \label{sub:bakgrund}
        Parans är utvecklare av en produkt som via optiska fibrer levererar naturligt solljus in i byggnader som ett alternativ till dagens traditionella ljuskällor. 
        Bolaget är baserat i Göteborg men levererar systemen globalt och har flera installationer runt om i världen. \bigskip

        Produkten fokuserar in solljus i optiska fibrer och styrs med hjälp av två stegmotorer för att följa solens bana. 
        Styrningen sker med en algoritm som ger solposition i grader, baserat på installationsplatsens geografiska position och tid, och för finare styrning av panelen då solen är framme inhämtas data från en ljussensor med fotocell.
        Detta för att alltid maximera det solljus som fokuseras in i fibern.\bigskip

        Styrkortet och motorerna till panelen drivs av en spänning om tolv (12) volt och kortet är en egen design kring mikrokontrollen PIC32. 
        Källkoden till panelen är skriven i \texttt{C} och kommunikation till enheten sker via seriell förbindelse över USB, där en USB till RS-232-omvandlare är integrerad på styrkortet. För att skicka instruktioner till panelen används en terminalemulator. \bigskip

        Ljussensorn som används i solpanelen kan representeras som ett koordinatsystem där sensorn som standard förväntar sig att ljuset genom en lins fokuseras till en punkt som träffar origo. 
        Parans problem är att vid tillverkning av panelen kan linsen fokusera ljuset något vid sidan av sensorns origo, vilket leder till sämre ljusintag till de optiska fibrerna.
        Efter fabriksmontering åtgärdas detta genom att kalibrera sensorn, genom att flytta punkten på det koordinatsystem som ljuset fokuseras ned till. Parans använder en manuell metod där man via en terminalemulator anger kommandon som vrider solpanelen och sedan kontrolleras värdet på en separat luxmätare. \bigskip

        Vid installation hos slutkund har det i ibland visat sig att en enstaka panelers kalibrering inte har varit korrekt och vad detta beror på är fortfarande inte klarlagt. Sensorn måste då kalibreras om på installationsplatsen, vilket idag vanligtvis sker manuellt av två personer.
        Den ena personen befinner sig då på taket vid panelen och justerar ljussensorns kalibreringsvärden, via den seriella anslutningen, samtidigt som den andra personen uppskattar den upplevda ljusstyrkan i det upplysta rummet. \bigskip
    % subsection bakgrund (end)

    \subsection{Syfte} % (fold)
    \label{sub:syfte}
          Syftet med projektet är att möjliggöra en helt automatisk process som kan kalibrera fotosensorn i Parans solpaneler så att maximalt ljusflöde från panelen kan uppnås. Syftet med processen är att minska tidsåtgången och höja precisionen jämfört med dagens manuella metod. 
          Vidare syftar projektet till att möjliggöra kommunikation mellan panelen och en luxmätare inne i byggnaden, så att kalibreringstekniken kan nyttjas till systemen generellt, oavsett om de är tagna i bruk eller i fabrik.
    % subsection syfte (end)

    \subsection{Mål} % (fold)
    \label{sub:mal}
        Målet med det här projektet är att ta fram en produkt som justerar fokuspunkten på ljussensorn, vilket då vrider på solpanelen för att lokalisera de x- och y-värden där intaget av solljus är som störst. Produkten ska kunna användas vid ett kalibreringstillfälle och sedermera avlägsnas från installationsplatsen.
        Ljusstyrkan mäts med hjälp av en luxmätare som levererar ljusintaget till en dator eller till en annan programmerbar enhet. 
        När det maximala ljusintaget är uppmätt registreras x- och y-värdena som den nya fokuspunkten för ljussensorn istället för det förinställda värdet på origo. 
        Vidare är målet med produkten att den ska stödja kommunikation mellan en luxmätare inne i byggnaden och en panel som befinner sig på taket. 
    % section mal (end)


    \subsection{Frågeställning} % (fold)
    \label{sub:fragestallning}
        \begin{itemize}
            \item Vilken algoritm kan anses vara lämplig för kalibreringen?
            \item Vilka förutsättningar för kommunikation finns mellan solpanelen och det upplysta rummet? 
            \item Hur tillförlitligt är det valda kommunikationssättet? 
            
        \end{itemize}
    % subsection fr_gest_llning (end)

    \subsection{Avgränsningar} % (fold)
    \label{sub:avgr_nsningar}
        \subsubsection{Hårdvara} % (fold)
        \label{ssub:h_rdvara}
            Redan existerande hårdvara kommer att användas, det vill säga sådan avsedd att användas för de ändamål nödvändiga för projektet. 
            Den primära hårdvaran, solpanel och luxmätare, kommer att tillhandahållas av uppdragsgivaren och inga alternativ till dessa kommer att undersökas. 
            Eventuell övrig hårdvara kan antingen vara helhetslösningar eller sådana som löser delproblem och kombineras. 
            De lösningar som kommer att undersökas och utvecklas är begränsade till att stödja företagets panel SP3.
        % subsection h_rdvara (end)

        \subsubsection{Mjukvara} % (fold)
        \label{ssub:mjukvara}
            Mjukvara kommer att utvecklas för att nå projektets uppsatta mål. 
            Denna kan komma att inkludera användning av både medföljande och externa ramverk och bibliotek för att lösa olika delproblem, exempelvis grafisk framställning och kommunikation mellan olika enheter.
        % subsection mjukvara (end)

    % section avgr_nsningar (end)

% section indroduktion (end)
\section{Metod} % (fold)
\label{sec:metod}
    
    \subsection{Vetenskaplig metod} % (fold)
    \label{sub:vetenskaplig_metod}
    	Detta projekt har tillämpat en variant av den vetenskapliga metoden Design Science Research (DSR). Metoden anses lämplig till problemlösande forskning där redan existerande produkter ska vidareutvecklas \cite{dsr}. Målet med DSR är att skapa artefakter, exempelvis en praktisk lösning, metod eller lösningsförslag, som löser de problem som identifierats inom projektet. \bigskip

        Design Science valdes då dess mål stämmer bra överens med det som projektet syftar till att göra. Detta kan sättas i kontrast med mer traditionella vetenskaper som snarare syftar till att utforska, förklara eller förutse fenomen \cite[s.~13]{dsr}. Att DSR valdes som metod över fallstudier eller 'action research' är återigen att målen överensstämmer med projektet, men även att typen av kunskap som anskaffas stämmer bättre överens än de andra två alternativen \cite[s.~95]{dsr}.\bigskip

    	Dresch et al. rekommenderar, baserat på studier av flera metoder för DSR, en metod i 12 steg \cite[s.~118--126]{dsr}. De tre inledande stegen är en analys av de problem som ska lösas, problemidentifiering, problemförståelse och litteraturstudier. Denna inledande fas mynnar ut i att hitta eventuella befintliga lösningar som kan vara lämpliga och att sedan föreslå en vidareutveckling och tillämpning av denna eller att föreslå en ny lösning. Steg sex till åtta är sedan att utforma, utveckla och utvärdera lösningen. Därefter ska den kunskap som givits av tidigare steg tydliggöras och slutsatser dras. Tidigarenämnda steg itereras vid behov för att uppnå önskat resultat. Slutligen ska generalisering av lösningen utformas och resultatet presenteras. \bigskip

    	Ovan nämnda metodik har för detta projekt förenklats något för att anpassas till projektets storlek och omfattning.
    % subsection vetenskaplig_metod (end)

    \subsection{Arbetsmetodik} % (fold)
    \label{sub:arbetsmetodik}
        Projektet arbetsmetodik utgick ifrån versionshanteringsverktyget 'git' 
        för den mjukvara som projektet använde sig av. För att få tillgång 
        till en central hantering av dokumenten använde sig projektet av 'GitHub.com' vilket även bistod med ett grafiskt gränssnitt till git, då git i sig själv endast har ett textbaserat gränssnitt. \bigskip

        Vidare var arbetsmetodiken inspirerad av 'Scrum' där större mål sattes upp och bröts ner till mindre så kallade 'issues' \cite{scrum}. Dessa issues sattes upp på en virtuell tavla med hjälp av verktyget 'Waffle.io' för att få en bättre överblick kring hur projektet utvecklades och vad som behövde göras. \bigskip

        Anledningen till att inte hela Scrum-metodiken anammades var att projektet utfördes av få personer så den rollfördelning som hör till i Scrum gick ej att utföra, samt att ovanan vid denna typ av utveckling gjorde att kostnaderna för varje issue var svårt att bestämma. Vidare var projektets omfång väl avgränsat av uppdrags\-givaren så dessa användes som milstenar istället för de föreslagna användarberättelserna \cite{scrum}. 

    % subsection arbetsmetodik (end)
% section metod (end)
\section{Teknisk bakgrund} % (fold)
\label{sec:teknisk_bakgrund}
    \subsection{Parans SP3} % (fold)
    \label{sub:parans_sp3}
        SP3 är tredje generationens solpanel utvecklad av Parans \cite{parans_manual}. Panelen monteras på utsidan av en byggnad, ofta på taket, och fokuserar solljus genom linser in i optisk fiber för att sedan genom armatur lysa upp inomhus. Varje panel har sex utgående kablar med fiberoptik, vardera ansluten till en armatur, vilkas räckvidd är upp till 20 meter. Två stegmotorer används för att justera panelens riktning horisontellt och vertikalt, styrda av ett mikrokontrollerkort, så att linserna alltid är vända mot solen. Motorernas rörelser bestäms av en algoritm i mjukvaran som räknar ut solens nuvarande position på himlen baserat på tid, datum och installationsplatsens geografiska position angivet i longitud och latitud som grader med sex decimaler. Mjukvaran som körs på mikrokontrollern är skriven i \texttt{C}.

        \subsubsection{Mikrokontrollerkortet} % (fold)
        \label{ssub:mikrokontrollerkortet}
            Mikrokontrollerkortet som används i panelen är formgivet av Parans och är baserat på en PIC32 mikrokontroller. PIC32 är en kategori mikrokontroller tillverkade av Microchip Technology för användning i inbyggda system och ger tillgång till bland annat flera I/O-anslutningar och UART för seriell kommunikation \cite[s.~3]{PIC32}. För att kommunicera med mikrokontrollerkortet med en dator finns en USB-port som ger en seriell anslutning som hanteras av en UART-krets från Silicon Laboratories, CP2102. Detta kräver att den anslutna datorn har en drivrutin för CP2102 installerad och möjliggör anslutning via en terminalemulator för installation, diagnostik och underhåll.
        % subsubsection mikrokontrollerkortet (end)

        \subsubsection{Fiberoptik} % (fold)
        \label{ssub:fiberoptik}
        % subsubsection fiberoptik (end)
        
    % subsection parans_sp3 (end)
% section teknisk_bakgrund (end)
\section{Genomförande} % (fold)
\label{sec:genomf_rande}

    Projektet genomfördes med stöd av den valda metoden. I problemidentifikations-fasen hölls möten med uppdragsgivaren i syfte att få en enhällig uppfattning om vad företaget efterfrågade och formaliserade det praktiska problem som företaget sökte en lösning till. Vidare är det även i den här fasen som frågeställningen utvecklades och fastslogs. \bigskip

    Under den andra fasen genomfördes litteratursökningar och diskussioner kring hur den tänkta lösningen skulle kunna utformas.  \bigskip

    I den tredje fasen identifierades två stycken artefakter som behöver utformas för att uppnå projektets mål, dels en algoritm som kan utföra själva kalibreringen och dels en enhet för att kommunicera över den optiska fiberkabeln. Projektet genomfördes i två steg, där algoritmen utvecklades först, då denna är av störst vikt enligt direktiv från företaget.
% section genomf_rande (end)
\section{Resultat} % (fold)
\label{sec:resultat}
    \subsection{Algoritm} % (fold)
    \label{sub:algoritm}
        För att möjliggöra automatisk kalibrering av panelens ljussensor har projektet utvecklat en sökalgoritm med positionsregistrering och stegreducering. Den utvecklade algoritmen gav svar på den första av projektets tre frågeställningar, ''Vilken algoritm kan anses vara lämplig för kalibreringen?''. Algoritmen uppsöker ett lokalt maximum i ljusstyrka, beskrivet i avsnitt~\ref{ssub:utveckling_av_algoritm}, och fanns lämplig för ändamålet.\bigskip

        Algoritmen söker stegvis efter det maximala inlästa värdet tills inga kringliggande större värden påträffas. I varje söksteg justeras panelens korrigeringsvärde för ljussensorn, vilket får panelen att vrida sig till den position som ger solljusets fokus i ljussensors korrigerade mittpunkt. Sökning sker i fyra riktningar, representerade av väderstrecksuttryck motsvarande den koordinatsystemsrepresentation panelen har för korrigeringsvärden där positiva x och y är öst respektive nord, och sker medurs med utgångsriktning österut. Om ett lika stort eller större värde avläses efter en vridning av panelen så kommer nästkommande undersökta position vara i samma riktning som den senast utförda, då avlästa värden antas vara kontinuerligt fallande från maximipunkten. Detta i enlighet med initialt antagande, representerat i figur~\ref{fig:array}, och senare undersökning, enligt figur~\ref{fig:yocto}. Algoritmen registrerar besökta positioner så att samma position ej undersöks upprepade gånger. Flödesschema för algoritmen finnes i bilaga \ref{sec:sokalgoritm_flow}. \bigskip

        Tillgången till kontinuerligt solljus är en förutsättning för kalibrering av enheter tagna i bruk då variationer i molnighet markant påverkar ljusintensiteten och således det avlästa värdet. I händelse av längre tids molnighet deaktiveras ljussensorn av panelens mjukvara och kalibrering går då ej att genomföra. Om så sker under pågående kalibrering återställs panelens korrigeringsvärden till de värden som var aktuella innan den automatiska kalibreringen startade. För att ytterligare motverka oförutsedda problem vid kalibreringstillfället har kontroller för timeout och korrigeringsvärdenas rimlighet implementerats, där båda kontrollerna vid utslag avbryter sökningen och korrigeringsvärdena återställs.

    % subsection algoritm (end)
    \subsection{Optisk kommunikation} % (fold)
    \label{sub:optisk_kommunikation}
        För att svara upp mot målet att ta fram en lösning för ''kommunikation mellan en luxmätare inne i byggnaden och en panel som befinner sig på taket'' besvaras i efterföljande frågeställningar för förutsättningar för kommunikation och hur tillförlitliga dessa lösningar kan anses vara.

        \subsubsection{Förutsättningar} % (fold)
        \label{sub:forutsattningar}
                
            Frågeställningen ''[v]ilka förutsättningar för kommunikation finns mellan solpanelen och det upplysta rummet'' har resulterat i en undersökning som visade att trådlös kommunikation inte är att anse som lämplig, utan den redan dragna optiska fiberkabeln är det kommunikationsmedia som bör nyttjas. Projektet föreslog två metoder för att nyttja fibern som databärare, där en metod innebar att skicka asynkron seriell data och en annan metod att två optiska fibrer kopplas samman för att returnera ljusintaget genom panelens linser och där uppmäta ljusstyrkan.\bigskip

            Den första lösningen som projektet föreslog var att en mikrokontroller inne i det upplysta rummet omvandlar utdata från en luxmätare till en optisk signal som sedan sänds seriellt, enligt standarden RS-232, upp till panelen för att där avkodas av en andra mottagande mikrokontroller. När signalerna skickas via fibrerna strålas ljuset ut ur solpanelens linser, vilket mottagaren då kan analysera. Mottagaren är monterad på panelen och har en ljuskänslig sensor som omvandlar de optiska signalerna till digitala. De digitala signalerna skickas från mottagaren vidare till den enhet som utför den algoritm som är avsedd att kalibrera ljussensorn. \bigskip

            Den andra lösningen var att, istället för att mäta upp ljusstyrkan i rummet, koppla ihop två stycken optiska fibrer från samma panel i det rum de är avsedda att upplysa, vilket då skickar ljusintaget tillbaka upp till panelen. Genom att täcka över de linser som förser den ena fiberkabeln med ljus kommer den andra fiberkabeln att skicka ut sitt ljusintag mot de nu täckta linserna. I detta förslag kan en luxmätare placeras i övertäckningsanordningen och där, via omvägen till det upplysta rummet och tillbaka, mäta upp hur mycket ljus panelen tar emot. Då luxmätaren nu befinner sig på panelen kan den direkt skicka sin data till den enhet som förväntas utföra algoritmen.

        % subsection förutsättningar (end)

        \subsubsection{Tillförlitlighet} % (fold)
        \label{sub:tillf_rlitlighet}
            För att nå ett svar på frågan ''[h]ur tillförlitligt är det valda kommunikationssättet?'' utreddes flera alternativ för att sända data. Genom att använda de fördragna fiberoptiska kablarna säkerställs att det ljus som skickas från rummet upp till panelen alltid kommer att levereras, förutsatt att ljuskällan är tillräckligt stark. Detta i skarp kontrast mot en trådlös lösning där flera lager betong mellan rummet och panelen inte är en osannolik företeelse, vilket då skulle resultera att signalerna under normala förutsättningar aldrig når panelen. \bigskip

            Huruvida de två föreslagna lösningarna för ljussändning är tillförlitliga beror på hur ljuset läses av uppe vid panelen. I förslaget med två mikrokontroller, där data sänds seriellt via RS-232, är tillförlitligheten lägre. Lösningen är känslig för störningar från bakgrundsljus så en tillförlitlig fästningsanordning för panelen behöver tillverkas. Vid försök i labbmiljö har data kunnat sändas med hjälp en vanlig lysdiod, driven av 5~V och 20~mA, via fiberkabel om 16 meter och där tolkats från en enskild fiber. Varje fiberkabel består av sex stycken fibrer där varje fiber är kopplad till en egen lins för att fokusera in solljuset. När data nu skickas nedifrån och upp kommer linsen att agera omvänt genom att omvandla fiberns fokuserade ljus till parallella strålar. Då ljuset nu är parallellt istället för fokuserat finns en risk att skillnaden i ljusintensitet mellan hög och låg inte är tillräckligt stor för att registreras av fotosensorn eller att motståndet inte ändras tillräckligt mycket. För att ha möjlighet att registrera förändringarna av ljusstyrka hade det varit idealt att ha en fokuseringslins till fotoresistorn och att resistorn hade varit innesluten i någon form av behållare som fästs över de sex linserna. \bigskip

            Det andra förslaget, där två fiberkablar kopplas samman för att skapa rundgång, är tillförlitligheten högre. Ingen behandlad data behöver kommuniceras utan endast rådata i form av ljusstyrka skickas ut genom linserna. Luxmätaren är inte beroende av snabba ändringar i ljuset, utan ljuset är konstant vilket leder till en högre tillförlitlighet. Ljusstyrkan kommer vara lägre när den kommer upp till panelen jämfört med om den skulle stråla ut i rummet, då den behöver färdas dubbelt så långt i fiberkabeln, men mätningen av ljusstyrkan är inte beroende av ett korrekt absolut värde. När panelen kalibreras är det istället av intresse att finna det högsta relativa värdet, det värde som uppmäts när panelen tar in mest ljus, då det är till det högsta värdet som panelen ska vara kalibrerad. Detta medför också att denna metod är mindre känslig för bakgrundsljus, så länge bakgrundsljuset är konstant, eftersom skillnaden i det utstrålade ljuset ändå kan registreras.\bigskip 

            För att kontrollera vilket luxvärde panelen faktiskt levererar till rummet behöver mer kvalificerad utrustning användas, utrustning som är kalibrerad och granskad för att mäta luxvärden i inomhusmiljö. Det är apparatur som företaget har tillgängligt men som ligger utanför detta projekt.
        % subsection tillf_rlitlighet (end)
    % subsection optisk_kommunikation (end)

    \subsection{Applikation} % (fold)
    \label{sub:applikation}
        Den framtagna applikationen gav en förutsättning att uppfylla målet ''att ta fram ett automatiskt system som justerar fokuspunkten på ljussensorn''. En automatisk kalibrering av panelens korrigeringsvärde för ljussensorn kan utföras genom att applikationen har kopplat samman sökalgoritmen med solpanelen och en luxmätare.\bigskip

        Sökalgoritmen implementerades i form av en Pythonapplikation med grafiskt gränssnitt och applikationen stödjer inhämtning av ljusvärden från en luxmätare, där värden förmedlas genom antingen direkt anslutning till datorn eller seriell kommunikation från en annan enhet. Den luxmätare som användes vid implementationen av direkt anslutning var Yocto-Light-V3 medan Adafruit TSL2591 användes för avläsning som överfördes seriellt via en Arduino Uno. Luxmätarna är beskrivna i avsnitt \ref{sub:yocto} respektive \ref{ssub:ada_tsl2591}. För seriell kommunikation använder applikationen sig av pySerial, ett bibliotek som kan hantera seriell kommunikation på de flesta vanligt förekommande operativsystem \cite{pyserial}. \bigskip

        Hos Parans fanns sedan tidigare kringutrustning som använder en Pythonapplikation med grafiskt gränssnitt anpassat till en pekskärm på en Raspberry Pi, där det grafiska gränssnittet är implementerat med ramverket TkInter \cite{solarremote}. Samma ramverk och grafiska formgivning har använts till den applikation som utvecklats av projektet. Detta upplägg är tänkt att underlätta framtida hantering och utveckling av applikationerna och möjliggör en eventuell framtida integrering av de båda. \bigskip

        Applikationen är utvecklad enligt en objektorienterad utvecklingsmodell och nya avläsningsmetoder kan implementeras utan större ingrepp i befintlig kod. För en översikt av källkodens struktur se UML diagrammen i bilaga~\ref{sec:uml_diagram}. Mjukvaran till Parans kommande solpanel, SP4, var ej färdigställd under projektets gång och således är applikationen riktad till SP3. Implementeringen av sökalgoritmen kan återanvändas till kommande versioner av solpaneler men vissa anpassningar kan behövas.
    % subsection applikation % (end)
% section resultat (end)

\section{Diskussion} % (fold)
\label{sec:diskussion}

    Projektet kan anses ha två huvudsakliga syften, där det första syftet är att ''ta fram en helt automatisk process som kan kalibrera fotosensorn i Parans solpaneler [~\dots~] med en lägre tidsåtgång och högre precision än dagens manuella metod''. 
    Den metod som företaget använde sig av tidigare, var dels baserad på manuell inmatning av värden, vilket tar tid och kan leda till fel på grund av den mänskliga faktorn och dels en manuell uppskattning av ljusstyrkan vilket även det kan leda till en felaktig kalibrering. 
    Med hjälp av den algoritm som projektet har utvecklat och redovisat, anser författarna att detta syfte är uppnått. Processen kan skötas helt automatiskt, så till vida att ljusflödet ut från panelen kan uppmätas. 
    Denna automatiserade kalibrering är att anses som tidsparande då inga värden behöver anges manuellt vilket sparar tid, särskilt då skillnaden mellan sensorns optimala inställningsvärde och det ursprungliga värdet är stort, så att många kalibrerings steg behöver göras. \bigskip

    Gällande bestämningen av ljusintensiteten finns det både för- och nackdelar mellan att göra en uppmätning av ljusstyrkan och en mänsklig uppskattning. Fördelarna med en automatiserad inläsning av ljusstyrkan är att kalibreringen blir standardiserad och inte behöver bero på personen som utför kalibreringen. När författarna deltog i ett praktiskt exempel av kalibrering av panelen ute i produktion upplevde vi att ljusintensiteten varierar väldigt mycket och med tanke på det mänskliga ögat och att dess anpassning till olika ljusintensiteter varierar olika snabbt beroende på om ljusintensiteten ökar eller minskar, kan kalibreringen tappa i precision vid en manuell bedömning.\cite[s.~273]{aot}  Det är svårt att jämföra hur två inställningar förhåller sig till varandra, vilken som är starkare eller svagare, om ljuskällan blivit väldigt mörk mellan de båda tillfällena. Det ska dock påpekas att en rent mekanisk bedömning har sina brister då ''[d]et är i det närmaste omöjligt att planera ljusmiljö [ \dots ] enbart med hjälp av fysikaliska mätningar'' men det påpekas också i litteraturen att det krävs erfarenhet för att kunna göra lämpliga bedömningar. \cite[s.~278]{aot} Personer med den erfarenheten finns sällan att tillgå för företaget då deras tekniker inte har möjlighet att befinna sig i rummet dit ljuset leder, utan befinner sig vid panelen för att sköta kalibreringen. Detta leder då till att antingen behöver teknikern gå mellan panelen och det upplysta rummet, något som är väldigt tidskrävande, alternativt krävs det två personer för att kalibreringen, en som sköter inmatningen till panelen och en som rapporterar ljusstyrkan. Sammantaget är vår bedömning att en uppmätning av ljusstyrkan är den mest lämpliga metoden då det sparar tid vid kalibreringen och blir oberoende på operatörers erfarenhet gällande bedömning av ljusintensitet.

% section diskussion (end)

\newpage
\printbibliography
\addcontentsline{toc}{section}{Referenser}

\newpage

\setcounter{page}{1}
\pagenumbering{Roman}
\appendix
\section*{Bilagor} % (fold)
\label{sec:bilagor}
\addcontentsline{toc}{section}{Bilagor}
\section{Simuleringsresultat av kalibreringsalgoritm} % (fold)
\label{sec:sokalgoritm_sim}
    \begin{table}[ht]
        % \caption{\label{tab:algoritm_steg}Jämförelse av algoritmernas antal söksteg}
        \centering
        \begin{threeparttable}
        \begin{tabular}{rccccccc}\toprule
            & \multicolumn{3}{c}{ $\mathscr{A}$: 8 riktningar} &  & \multicolumn{3}{c}{ $\mathscr{B}$: 4 riktingar} \\ %\cline{2-4} \cline{6-8} 
            \hspace{5mm}    Steg        & v.1   & v.2   & v3    & \hspace{5mm}  & v.1   & v.2   & v3    \\  \midrule
                            Medel       & 255,7 & 134,2 & 78,8  &               & 160,2 & 130,5 & 72,0  \\
                            Median      & 215   & 124   & 76    &               & 140   & 120   & 69    \\
                            Max         & 1065  & 450   & 195   &               & 621   & 448   & 186   \\
                            Min         & 1     & 7     & 9     &               & 1     & 4     & 5     \\
                            $\sigma$    & 176.3 & 73,5  & 33,1  &               & 101,1 & 73.4  & 32,9  \\  \bottomrule
        \end{tabular}
        \begin{tablenotes}
            \item Varje version implementerar ny funktionalitet:
            \item 1. Söker medurs i åtta alternativt fyra riktningar.
            \item 2. Testa ej redan kontrollerade koordinater.
            \item 3. Testa först den riktning som senast lyckad förflyttning.
        \end{tablenotes}
        \end{threeparttable}
    \end{table} \bigskip

%                 v3:8        v3:4        v2:8                   v2:4            v1:8            v1:4
% steps   max     195         186          450                   448             1065             621
%         min       9           5            7                     4                1               1
%         mean     78.7549     72.0076     134.15                130.483          255.7382        160.1794
%         median   76.0        69.0        124.0                 120.0            215.0           140.0

% std dev          33.063      32.918       73.486                73.386          176.294         101.092

\documentclass[a4paper, 11pt]{article}

%%--LANGUAGE AND ENCODING--%%
\usepackage[swedish]{babel}
\usepackage[utf8]{inputenc}

%%--SANS-SERIF FONTS FOR SECTIONS--%%
\usepackage{sectsty}
\usepackage{helvet}
\allsectionsfont{\bfseries\sffamily}

%%--GRAPHICS--%%  (Requires preamble)
\usepackage{tikz}
\usetikzlibrary{backgrounds,calc,positioning,shapes,arrows}

% % Declaration of different types of figures for use in flow charts
\tikzstyle{startstop} = [rectangle, rounded corners, minimum width=3cm, minimum height=1cm,text centered, draw=black, fill=red!30]
\tikzstyle{io} = [trapezium, trapezium left angle=70, trapezium right angle=110, minimum width=2cm, minimum height=1cm, text centered, draw=black, fill=blue!30]
\tikzstyle{process} = [rectangle, minimum width=3cm, minimum height=1cm, text centered, draw=black, fill=orange!30]
\tikzstyle{decision} = [diamond, minimum width=3cm, minimum height=1cm, text centered, draw=black, fill=green!30]
\tikzstyle{arrow} = [draw, thick,-latex',>=stealth]


\begin{document}
\section*{Sökalgoritm} % (fold)
\label{sec:sokalgoritm}

\begin{tikzpicture}[node distance=2cm]
    \tikzstyle{every node}=[font=\scriptsize]
    \node (start) [startstop] {Start};
    \node (inp1) [io, below=5mm of start, text width=20mm] {Läs position och värde};
    \node (success) [decision, below=5mm of inp1, text width=20mm] {Nytt max funnet?};
    \node (iterate) [process, right of=success, text width=25mm, xshift=3cm] {Iterera riktningar (E, S, W, N)};
    \node (prev1) [decision, below=56mm of iterate, text width=20mm] {Tidigare undersökt?};
    \node (allprev) [decision, below=20mm of prev1, text width=20mm] {Alla riktningar kontrollerade?};
    \node (same) [process, below=5mm of success, text width=25mm] {Fortsätt i \hbox{samma} riktning};
    \node (prev2) [decision, below=5mm of same, text width=20mm] {Tidigare undersökt?};
    \node (inp2) [io, below=5mm of prev2, text width=20mm] {Läs \hbox{aktuellt} värde};
    \node (greater) [decision, below=5mm of inp2, text width=20mm] {Nytt max funnet?};
    \node (update) [process, below=5mm of greater, text width=25mm] {Uppdatera max och position};

    \node (stop) [startstop, below=5mm of allprev] {Stopp};

    \draw [arrow] (start) -- (inp1);
    \draw [arrow] (inp1) -- (success);
    \draw [arrow] (success) -- node[anchor=south] {Nej} (iterate);
    \draw [arrow] (success) -- node[anchor=east] {Ja} (same);
    \draw [arrow] (same) -- (prev2);
    \draw [arrow] (iterate) -- (prev1);
    \draw [arrow] (prev1) -- node[anchor=east] {Ja} (allprev);
    % \draw [arrow] (prev1) -- node[anchor=south] {Nej} +(-2.25,0) |- (inp2);
    \draw [arrow] (prev1) -- node[anchor=south] {Nej} (inp2);
    \draw [arrow] (allprev) -- node[anchor=east] {Ja} (stop);
    \draw [arrow] (allprev) -- node[anchor=south] {Nej} +(3,0) |- (iterate);
    \draw [arrow] (prev2) -- node[anchor=east] {Nej} (inp2);

    \draw [arrow] (inp2) -- (greater);

    \draw [arrow] (greater) -- node[anchor=east] {Ja} (update);

    \draw [draw, thick] (prev2) -- node[anchor=south] {Ja} +(-3,0);
    \draw [draw, thick] (greater) -- node[anchor=south] {Nej} +(-3,0);

    \draw [arrow] (update) -- +(-3,0) |- (success);

\end{tikzpicture}
% section sokalgoritm (end)
\end{document}
\section{Specifikationer} % (fold)
\label{sec:specifikationer}

    \subsection{Arduino} % (fold)
    \label{sub:arduino_spec}
        \begin{tabularx}{\textwidth}{@{}lX}

            Microcontroller & ATmega328 \\
            Operating Voltage &  5V \\
            Input Voltage (recommended) & 7-12V \\
            Input Voltage (limits) & 6-20V \\
            Digital I/O Pins  &  14 (of which 6 provide PWM output) \\
            Analog Input Pins  & 6 \\
            DC Current per I/O Pin  & 40 mA \\
            DC Current for 3.3V Pin & 50 mA \\
            Flash Memory  &  32 KB (ATmega328) of which 0.5 KB used by bootloader \\
            SRAM  &  2 KB (ATmega328) \\
            EEPROM & 1 KB (ATmega328) \\
            Clock Speed & 16 MHz \\
            Length & 68.6 mm \\
            Width  & 53.4 mm \\
            Weight & 25 g     \\
        \end{tabularx}
    % subsection arduino_spec (end)

    \subsection{Sändare} % (fold)
    \label{sub:sandare}
    \begin{tabularx}{\textwidth}{@{}lX}
        Mikrokontroller & Arduino Uno \\
        Lysdiod & 20\thinspace000 candela \\
        Resistor & 100 $\Omega$
        
    \end{tabularx}
    % subsection s_ndare (end)

    \subsection{Mottagare} % (fold)
    \label{sub:mottagare}
      \begin{tabularx}{\textwidth}{@{}lX}
        Mikrokontroller & Arduino Uno \\
        Fotoresistor & 18 - 50 M$\Omega$ \\
        Resistor & 10\thinspace000 $\Omega$
        
    \end{tabularx}
    % subsection mottagare (end)

% section specifikationer (end)
\begin{figure}[hbt]
                \pgfplotstabletypeset[color cells={min=20.0 , max=33.0}, /pgfplots/colormap={yellowred}{rgb255(0cm)=(255,255,105); rgb255(2cm)=(255,10,10)},]
                {
                7.70     12.3    16.9    18.7    21.3    21.5    23.0
                11.7    14.9    18.2    23.4    26.8    26.1    29.3 
                14.9    19.3    25.2    29.0    32.5    32.6    32.2 
                18.5    22.7    30.6    32.7    31.6    31.1    29.9 
                23.1    25.5    24.5    31.1    27.9    27.6    25.8 
                27.3    29.8    31.3    27.7    24.9    24.3    22.0
                29.3    27.0    29.6    25.3    19.8    20.7    18.8 
                26.1    29.6    27.4    23.2    18.0    18.2    16.3 
                30.5    32.2    29.6    26.6    23.0    20.0    17.9 
                }
\end{figure}









\end{document}
