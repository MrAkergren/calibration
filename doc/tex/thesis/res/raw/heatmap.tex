% \documentclass[a4paper]{article}

% \usepackage{colortbl}
% \usepackage{pgfplots}
% \usepackage{pgfplotstable}
% \usepackage{graphicx}
% \pgfplotsset{compat=1.8}
% \pgfplotstableset{
%     every head row/.style={ 
%         output empty row,
%     },
%     /color cells/min/.initial=0,
%     /color cells/max/.initial=1000,
%     /color cells/textcolor/.initial=,
%     %
%     % Usage: 'color cells={min=<value which is mapped to lowest color>, 
%     %   max = <value which is mapped to largest>}
%     color cells/.code={%
%         \pgfqkeys{/color cells}{#1}%
%         \pgfkeysalso{%
%             postproc cell content/.code={%
%                 %
%                 \begingroup
%                 %
%                 % acquire the value before any number printer changed
%                 % it:
%                 \pgfkeysgetvalue{/pgfplots/table/@preprocessed cell content}\value
%                 \ifx\value\empty
%                     \endgroup
%                 \else
%                 \pgfmathfloatparsenumber{\value}%
%                 \pgfmathfloattofixed{\pgfmathresult}%
%                 \let\value=\pgfmathresult
%                 %
%                 % map that value:
%                 \pgfplotscolormapaccess
%                     [\pgfkeysvalueof{/color cells/min}:\pgfkeysvalueof{/color cells/max}]
%                     {\value}
%                     {\pgfkeysvalueof{/pgfplots/colormap name}}%
%                 % now, \pgfmathresult contains {<R>,<G>,<B>}
%                 % 
%                 % acquire the value AFTER any preprocessor or
%                 % typesetter (like number printer) worked on it:
%                 \pgfkeysgetvalue{/pgfplots/table/@cell content}\typesetvalue
%                 \pgfkeysgetvalue{/color cells/textcolor}\textcolorvalue
%                 %
%                 % tex-expansion control
%                 % see http://tex.stackexchange.com/questions/12668/where-do-i-start-latex-programming/27589#27589
%                 \toks0=\expandafter{\typesetvalue}%
%                 \xdef\temp{%
%                     \noexpand\pgfkeysalso{%
%                         @cell content={%
%                             \noexpand\cellcolor[rgb]{\pgfmathresult}%
%                             \noexpand\definecolor{mapped color}{rgb}{\pgfmathresult}%
%                             \ifx\textcolorvalue\empty
%                             \else
%                                 \noexpand\color{\textcolorvalue}%
%                             \fi
%                             \the\toks0 %
%                         }%
%                     }%
%                 }%
%                 \endgroup
%                 \temp
%                 \fi
%             }%
%         }%
%     }
% }

% \begin{document}
\section{Uppmätt fokuspunkt från SP3} % (fold)
\label{sec:heatmap}


    
\begin{figure}[hbt]

\setlength\tabcolsep{1 pt}
\def\arraystretch{2}
 \centering
                \pgfplotstabletypeset[color cells={min=48.0 , max=110.0}, /pgfplots/colormap={yellowred}{rgb255(0cm)=(255,255,105); rgb255(2cm)=(255,10,10)},font=\tiny,
                     begin table = \begin{tabular}{>{\centering}p{0.5cm}>{\centering}p{0.5cm}>{\centering}p{0.5cm}>{\centering}p{0.5cm}>{\centering}p{0.5cm}>{\centering}p{0.5cm}>{\centering}p{0.5cm}>{\centering}p{0.5cm}>{\centering}p{0.5cm}>{\centering}p{0.5cm}>{\centering}p{0.5cm}>{\centering}p{0.5cm}>{\centering}p{0.5cm}>{\centering}p{0.5cm}>{\centering}p{0.5cm}>{\centering}p{0.5cm}>{\centering}p{0.5cm}>{\centering}p{0.5cm}>{\centering}p{0.5cm}p{0.5cm}},
    end table = \end{tabular},]
                {
5  6  7  9  9  10  12  13  14  15  16  16  15  15  14  13  11  9  8  7  
7  8  10  11  13  14  17  18  18  19  20  21  21  21  20  18  16  13  12  11  
9  11  13  14  17  20  21  24  25  28  30  32  31  30  28  24  23  20  16  15  
11  13  16  20  22  26  31  35  38  42  49  50  45  44  40  36  39  30  24  19  
14  17  22  26  32  39  44  52  58  67  76  77  73  74  71  60  60  47  38  27  
17  24  31  38  47  59  70  82  92  97  95  94  95  97  98  96  85  71  55  42  
20  34  45  60  71  83  99  98  96  94  93  92  93  95  97  98  90  81  66  54  
25  39  50  62  74  90  99  97  94  92  90  90  90  92  95  98  94  83  70  57  
31  40  52  64  77  94  99  96  92  89  87  88  87  89  92  96  99  89  74  60  
40  42  55  69  82  100  98  94  89  85  81  82  81  85  87  92  98  95  78  63  
42  46  61  78  91  99  95  90  85  78  73  71  71  73  77  87  94  99  89  70  
44  52  69  85  99  97  91  82  75  71  65  61  61  63  68  78  88  97  99  77  
45  57  76  93  98  93  84  75  69  64  59  57  55  58  63  70  82  94  99  87  
48  62  81  99  97  90  81  71  65  60  55  54  52  54  58  66  77  89  98  93  
49  64  82  99  96  89  78  69  63  58  54  52  51  53  56  63  75  88  98  96  
51  63  82  99  96  89  79  71  63  58  55  52  52  54  57  64  75  88  98  94  
51  61  80  100  96  91  81  72  65  60  56  54  53  55  58  65  76  88  98  95  
48  59  76  97  97  92  83  73  66  61  56  54  53  54  57  63  73  86  97  99  
46  49  66  84  99  94  91  95  89  85  84  84  82  82  85  92  89  87  96  93  
43  9  10  12  14  15  17  20  21  22  22  24  25  23  22  20  18  16  13  11

                }
\end{figure}

Tabell över normerade värden uppmätta från Parans solpanel SP3. Värderna utgör procent från det maximalt uppmätta värdet, avrundat till närmsta heltal. Anledningen till normeringen är att  visa på storleksändring mellan talen, då de faktiska värderna är intetsägande då de inte direkt kan representera lux, på grund av felmarginal av luxmätaren och uppmätningsmetoden.
% section uppm_tt_fokuspunkt_fr_n_sp3 (end)
% \end{document}