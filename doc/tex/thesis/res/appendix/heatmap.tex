\documentclass[a4paper]{article}

\usepackage{colortbl}
\usepackage{pgfplots}
\usepackage{pgfplotstable}
\usepackage{graphicx}
\pgfplotsset{compat=1.8}
\pgfplotstableset{
    every head row/.style={ 
        output empty row,
    },
    /color cells/min/.initial=0,
    /color cells/max/.initial=1000,
    /color cells/textcolor/.initial=,
    %
    % Usage: 'color cells={min=<value which is mapped to lowest color>, 
    %   max = <value which is mapped to largest>}
    color cells/.code={%
        \pgfqkeys{/color cells}{#1}%
        \pgfkeysalso{%
            postproc cell content/.code={%
                %
                \begingroup
                %
                % acquire the value before any number printer changed
                % it:
                \pgfkeysgetvalue{/pgfplots/table/@preprocessed cell content}\value
                \ifx\value\empty
                    \endgroup
                \else
                \pgfmathfloatparsenumber{\value}%
                \pgfmathfloattofixed{\pgfmathresult}%
                \let\value=\pgfmathresult
                %
                % map that value:
                \pgfplotscolormapaccess
                    [\pgfkeysvalueof{/color cells/min}:\pgfkeysvalueof{/color cells/max}]
                    {\value}
                    {\pgfkeysvalueof{/pgfplots/colormap name}}%
                % now, \pgfmathresult contains {<R>,<G>,<B>}
                % 
                % acquire the value AFTER any preprocessor or
                % typesetter (like number printer) worked on it:
                \pgfkeysgetvalue{/pgfplots/table/@cell content}\typesetvalue
                \pgfkeysgetvalue{/color cells/textcolor}\textcolorvalue
                %
                % tex-expansion control
                % see http://tex.stackexchange.com/questions/12668/where-do-i-start-latex-programming/27589#27589
                \toks0=\expandafter{\typesetvalue}%
                \xdef\temp{%
                    \noexpand\pgfkeysalso{%
                        @cell content={%
                            \noexpand\cellcolor[rgb]{\pgfmathresult}%
                            \noexpand\definecolor{mapped color}{rgb}{\pgfmathresult}%
                            \ifx\textcolorvalue\empty
                            \else
                                \noexpand\color{\textcolorvalue}%
                            \fi
                            \the\toks0 %
                        }%
                    }%
                }%
                \endgroup
                \temp
                \fi
            }%
        }%
    }
}

\begin{document}
\section{Uppmätt fokuspunkt från SP3} % (fold)
\label{sec:heatmap}


    
\begin{figure}[hbt]

\setlength\tabcolsep{1 pt}
\def\arraystretch{2}
\centering
    \pgfplotstabletypeset[color cells={min=48.0 , max=110.0}, /pgfplots/colormap={yellowred}{rgb255(0cm)=(255,255,105); rgb255(2cm)=(255,10,10)},
    font=\tiny,
   % begin table = \begin{tabular}{>{\centering}p{0.5cm}>{\centering}p{0.5cm}
            % >{\centering}p{0.5cm}>{\centering}p{0.5cm}>{\centering}p{0.5cm}>{\centering}p{0.5cm}>{\centering}p{0.5cm}>{\centering}p{0.5cm}>{\centering}p{0.5cm}>{\centering}p{0.5cm}>{\centering}p{0.5cm}>{\centering}p{0.5cm}>{\centering}p{0.5cm}>{\centering}p{0.5cm}>{\centering}p{0.5cm}>{\centering}p{0.5cm}>{\centering}p{0.5cm}>{\centering}p{0.5cm}>{\centering}p{0.5cm}p{0.5cm}},
%    end table = \end{tabular}
]
{
3  4  5  6  7  8  9  9  9  11  10  7  10  9  9  9  8  7  9  6  
5  6  8  8  10  10  13  12  13  14  14  11  15  14  13  12  11  10  13  9  
6  8  10  11  13  14  16  17  18  19  21  16  24  21  19  18  16  14  17  11  
8  10  13  15  18  19  23  24  26  30  32  23  36  30  28  26  22  19  22  14  
11  13  17  20  25  27  31  34  40  43  44  36  52  42  36  38  31  27  29  19  
14  17  23  27  35  39  45  49  53  62  61  54  68  57  51  49  41  33  36  25  
17  23  30  36  44  51  58  67  71  79  78  74  71  72  66  61  51  43  46  33  
22  28  37  45  56  62  70  77  83  86  89  85  72  83  77  72  61  52  52  43  
25  30  38  47  55  65  71  78  84  86  89  89  74  84  83  74  67  57  51  50  
26  32  39  48  58  67  72  79  86  87  91  90  76  85  84  76  68  58  52  52  
27  31  39  49  59  68  74  80  87  89  92  91  79  86  85  78  70  59  53  53  
28  32  41  50  61  69  76  82  89  91  94  93  86  87  86  79  71  61  55  52  
28  34  44  52  63  73  80  86  92  94  96  95  90  91  87  82  73  63  58  55  
28  36  46  55  66  77  83  90  95  97  98  97  93  93  91  84  76  65  59  57  
30  37  47  57  67  76  85  91  96  99  100  99  94  94  91  85  76  67  59  57  
30  37  46  57  67  76  84  91  97  98  100  99  95  95  91  85  78  66  58  56  
29  35  45  56  66  76  82  90  96  97  99  98  95  93  89  83  76  64  54  53  
28  33  43  52  64  73  79  87  92  95  96  95  94  91  86  80  72  61  52  49  
27  32  41  50  59  71  77  84  88  93  94  91  93  88  82  75  68  59  49  47  
26  31  40  49  59  71  76  84  87  92  93  93  93  88  80  75  68  58  49  46  
 

}
\end{figure}

Tabell över normerade värden uppmätta från Parans solpanel SP3. Värderna utgör procent från det maximalt uppmätta värdet, avrundat till närmsta heltal. Anledningen till normeringen är att  visa på storleksändring mellan talen, då de faktiska värderna är intetsägande då de inte direkt kan representera lux, på grund av felmarginal av luxmätaren och uppmätningsmetoden.
% section uppm_tt_fokuspunkt_fr_n_sp3 (end)
\end{document}