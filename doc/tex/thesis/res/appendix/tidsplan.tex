\section{Tidsplan}
\label{sec:tidsplan} % (fold)
    Innan projektets start upprättades en tidsplan enligt följande. \bigskip

    Projektarbetet skulle inledas vecka 13 och målet var att, om inget oförutsett inträffade, den skriftliga rapporten skulle slutföras under vecka 23. Resultatet skulle sedan muntligen presenteras under vecka 24 eller 25. De första 2-3 veckorna planerades att mestadels ägnas åt planering, litteraturstudier och problemanalys. Därefter beräknades tiden ägnas åt utveckling, kompletterande informationsinhämtning och kontinuerlig utvärdering. Parallellt med detta skulle löpande rapportskrivning att ske. I projektets avslutande 1-2 veckor skulle större fokus ligga på att färdigställa den skriftliga rapporten och presentationen. \bigskip


    \begin{table}[h]
        \centering
        \begin{tabular}{@{}>{\centering}p{10mm}>{\centering}p{14mm}p{90mm}@{}}
            \toprule
            Vecka \newline (2015)        & \raggedright Projekt-vecka & Mål \\ \midrule
            13 & 1  & Planering och upprättande av arbetsstruktur, \newline uppstartsmöten \\
            14 & 2  & Litteraturstudier, planering och problemanalys \\
            15 & -- & Påskuppehåll \\
            16 & 3  & Litteraturstudier, problemanalys och utveckling \\
            17 & 4  & Utveckling \\
            18 & 5  & Utveckling och utvärdering \\
            19 & 6  & Utveckling och utvärdering \\
            20 & 7  & Utveckling och utvärdering \\
            21 & 8  & Utveckling och utvärdering \\
            22 & 9  & Utveckling, utvärdering och fokus på skriftlig rapport \\
            23 & 10 & Fokus på skriftlig rapport och presentation \\ \bottomrule
        \end{tabular}
    \end{table}

    
    % Below: Table used in original planning report

    % \noindent \begin{tabularx}{\textwidth}{@{}>{\centering}p{10mm}>{\centering}p{16mm}X}
    %     \textbf{Vecka \newline (2015)} & \raggedright\textbf{Projekt- vecka} & \textbf{Mål}\\ \hline
    %     13 & 1  & Planering och upprättande av arbetsstruktur, \newline uppstartsmöten \\
    %     14 & 2  & Litteraturstudier, planering och problemanalys \\
    %     15 & -- & Påskuppehåll \\
    %     16 & 3  & Litteraturstudier, problemanalys och utveckling \\
    %     17 & 4  & Utveckling \\
    %     18 & 5  & Utveckling och utvärdering \\
    %     19 & 6  & Utveckling och utvärdering \\
    %     20 & 7  & Utveckling och utvärdering \\
    %     21 & 8  & Utveckling och utvärdering \\
    %     22 & 9  & Utveckling, utvärdering och fokus på skriftlig rapport \\
    %     23 & 10 & Fokus på skriftlig rapport och presentation \\
    % \end{tabularx}

% section tidsplan (end)