\section{Genomförande} % (fold)
\label{sec:genomf_rande}

    Projektet genomfördes med stöd av den valda metoden. I problemidentifikations-fasen, fas ett, hölls möten med uppdragsgivaren i syfte att få en enhällig uppfattning om vad företaget efterfrågade och formaliserade det praktiska problem som företaget sökte en lösning till. Vidare är det även i den här fasen som frågeställningen utvecklades och fastslogs. \bigskip

    Under den första fasen genomfördes även litteratursökningar och diskussioner kring hur den tänkta lösningen skulle kunna utformas, vilket resulterade i att projektet kommer att utföras i två iterationer, en för algoritmen och en för kommunikationen. \bigskip

    Förutsättningen vid litteraturstudien, gällande kommunikationen mellan taket och byggnadens innandöme, var att den trådlösa kommunikationen skall ske med standardiserade protokoll. Detta för att underlätta mottagandet av den trådlösa sändningen, i syfte att undvika tidssänken i felsökning då projektet har en relativt snäv tidsram.\bigskip

    Det visade sig att trådlösa standarder för datakommunikation så som 802.11 standarderna har problem att sända när betongkonstruktioner hindrar utspridningen av radiovågorna och kräver speciell apparatur för att klara av att skicka data igenom sådana förhållanden \cite{11n}. Detta medför att trådlös kommunikation inte är lämplig för företaget, då de på förhand inte kan veta ifall deras kommunikation kommer att fungera på plats hos deras kunder. \bigskip

    Ett lämpligare medium att kommunicera via är istället de fiberoptiska kablar som redan är dragna, då rummet lyses upp av just dessa kablar. Enligt företaget kommer det finnas mer än en fiberkabel dragen till varje rum, vilket öppnar upp för möjligheten att koppla in apparatur för kommunikation i en fiberkabel, medan den eller de andra kablarna kan fortsätta hämta in ljus till rummet. Med de svårigheter som den trådlösa kommunikationen medförde i kombination med att ett fungerande alternativt medium redan finns draget, valde projektet att fokusera på det senare. \bigskip

    För kalibreringsalgoritmens del undersöktes vilka typer av datastrukturer som skulle komma att beröras. När problemet analyserades under problemförståelse fasen insågs att värdena som samlas in kan representeras som en matris (eng. array) där det finns ett unikt maxvärde och kring detta minskande värden som blir lägre ju längre från max värdet befinner sig.

    \texttt{INSERT MORE TEXT ABOUT ALGORITHM HERE} \bigskip

    I den andra fasen identifierades och presenterades två stycken artefakter som behöver utformas för att uppnå projektets mål, dels en algoritm som kan utföra själva kalibreringen och dels en enhet för att kommunicera mellan rummet och taket. Projektet genomfördes i två steg, där kommunikationen 
    utvecklades först dels då denna tillämpning krävde mer efterforskning och dels då viktig hårdvara saknades för testning av algoritmen. \bigskip

    Genom att fokusera på det optiska alternativet leder detta in projektet till det tredje steget i metoden, att föreslå en artefakt som löser det ställda problemet. Projektet föreslår då en lösning med en omvandlare från luxmätarens utdata till en optisk signal som sänds upp till taket för att där avkodas. Omvandlaren kan vara någon form av mikrokontroller så som till exempel en Arduino. Uppe på taken kan avkodaren även den vara en mikrokontroller, eller om det finns någon typ av ljussensor som direkt kan skicka sin data över USB till den programmerbara enhet som utför den algoritm som utvecklats.

% section genomf_rande (end)