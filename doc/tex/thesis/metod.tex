\section{Metod} % (fold)
\label{sec:metod}
    
    \subsection{Vetenskaplig metod} % (fold)
    \label{sub:vetenskaplig_metod}
    	Detta projekt har tillämpat en variant av den vetenskapliga metoden Design Science Research (DSR). Metoden anses lämplig till problemlösande forskning där redan existerande produkter ska vidareutvecklas \cite{dsr}. Målet med DSR är att skapa artefakter, exempelvis en praktisk lösning, metod eller lösningsförslag, som löser de problem som identifierats inom projektet. \bigskip

        Design Science valdes då dess mål stämmer bra överens med det som projektet syftar till att göra. Detta kan sättas i kontrast med mer traditionella vetenskaper som snarare syftar till att utforska, förklara eller förutse fenomen \cite[s.~13]{dsr}. Att DSR valdes som metod över fallstudier eller 'action research' är återigen att målen överensstämmer med projektet, men även att typen av kunskap som anskaffas stämmer bättre överens än de andra två alternativen \cite[s.~95]{dsr}.\bigskip

    	Dresch et al. rekommenderar, baserat på studier av flera metoder för DSR, en metod i 12 steg \cite[s.~118--126]{dsr}. De tre inledande stegen är en analys av de problem som ska lösas, problemidentifiering, problemförståelse och litteraturstudier. Denna inledande fas mynnar ut i att hitta eventuella befintliga lösningar som kan vara lämpliga och att sedan föreslå en vidareutveckling och tillämpning av denna eller att föreslå en ny lösning. Steg sex till åtta är sedan att utforma, utveckla och utvärdera lösningen. Därefter ska den kunskap som givits av tidigare steg tydliggöras och slutsatser dras. Tidigarenämnda steg itereras vid behov för att uppnå önskat resultat. Slutligen ska generalisering av lösningen utformas och resultatet presenteras. \bigskip

    	Ovan nämnda metodik har för detta projekt förenklats något för att anpassas till projektets storlek och omfattning.
    % subsection vetenskaplig_metod (end)

    \subsection{Arbetsmetodik} % (fold)
    \label{sub:arbetsmetodik}
        Projektet arbetsmetodik utgick ifrån versionshanteringsverktyget 'git' 
        för den mjukvara som projektet använde sig av. För att få tillgång 
        till en central hantering av dokumenten använde sig projektet av 'GitHub.com' vilket även bistod med ett grafiskt gränssnitt till git, då git i sig själv endast har ett textbaserat gränssnitt. \bigskip

        Vidare var arbetsmetodiken inspirerad av 'Scrum' där större mål sattes upp och bröts ner till mindre så kallade 'issues' \cite{scrum}. Dessa issues sattes upp på en virtuell tavla med hjälp av verktyget 'Waffle.io' för att få en bättre överblick kring hur projektet utvecklades och vad som behövde göras. \bigskip

        Anledningen till att inte hela Scrum-metodiken anammades var att projektet utfördes av få personer så den rollfördelning som hör till i Scrum gick ej att utföra, samt att ovanan vid denna typ av utveckling gjorde att kostnaderna för varje issue var svårt att bestämma. Vidare var projektets omfång väl avgränsat av uppdrags\-givaren så dessa användes som milstenar istället för de föreslagna användarberättelserna \cite{scrum}. 

    % subsection arbetsmetodik (end)
% section metod (end)