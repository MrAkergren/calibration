\section{Metod} % (fold)
\label{sec:metod}
	Detta projekt har tillämpat en variant av den vetenskapliga metoden Design Science Research (DSR). Metoden anses lämplig till problemlösande forskning där redan existerande produkter ska vidareutvecklas \cite{dsr}. Målet med DSR är att skapa artefakter, exempelvis en praktisk lösning, metod eller lösningsförslag, som löser de problem som identifierats inom projektet. \bigskip

	Dresch et al. rekommenderar, baserat på studier av flera metoder för DSR, en metod i 12 steg \cite[s.~118--126]{dsr}. De tre inledande stegen är en analys av de problem som ska lösas, problemidentifiering, problemförståelse och litteraturstudier. Denna inledande fas mynnar ut i att hitta eventuella befintliga lösningar som kan vara lämpliga och att sedan föreslå en vidareutveckling och tillämpning av denna eller att föreslå en ny lösning. Steg sex till åtta är sedan att utforma, utveckla och utvärdera lösningen. Därefter ska den kunskap som givits av tidigare steg tydliggöras och slutsatser dras. Tidigarenämnda steg itereras vid behov för att uppnå önskat resultat. Slutligen ska generalisering av lösningen utformas och resultatet presenteras. \bigskip

	Ovan nämnda metodik har för detta projekt förenklats något för att anpassas till projektets storlek och omfattning.
% section metod (end)