\section{Introduktion} % (fold)
\label{sec:indroduktion}

    \subsection{Bakgrund} % (fold)
    \label{sub:bakgrund}
        Parans är utvecklare av en produkt som via optiska fibrer levererar naturligt solljus in i byggnader, som ett alternativ till dagens traditionella ljuskällor. 
        Bolaget är baserat i Göteborg men levererar systemen globalt och har flera installationer runt om i världen. \bigskip

        Produkten fokuserar in solljus i optiska fibrer och styrs med hjälp av två stegmotorer för att följa solens bana. 
        Styrningen sker dels med en algoritm som, baserat på geografisk position och tid, ger en solposition i grader och dels från en solsensor med fotocell som ger data för en finstyrning av panelens positionering då solen är framme.
        Detta för att alltid maximera solljuset som levereras in i fibern.\bigskip

        Styrkortet och motorerna till panelen drivs av en spänning om tolv (12) volt och kortet är en egen design kring mikrokontrollen PIC32. 
        Källkoden till panelen är skriven i \texttt{C} och kommunikation till enheten sker via seriell förbindelse över USB, där en USB till RS-232 omvandlare är integrerad på styrkortet. För att skicka instruktioner till panelen används en terminalemulator. \bigskip

        Fotosensorn som används i solpanelen kan representeras som ett koordinatsystem, där sensorn förväntar sig att ljuset fokuseras till en punkt som träffar origo som standard. 
        Problemet som Parans har är tvådelat, det första problemet att i tillverkning av panelen kan linsen fokusera ner ljuset något vid sidan av origo på sensorn, vilket leder till sämre ljusintag i de optiska fibrerna. 
        Det andra problemet är att solen inte går att fokusera ner till en punkt, utan kommer alltid att representeras av en disk, vilket kan förvirra sensorn något och då även detta leda till sämre ljusintag i de optiska fibrerna. \bigskip

        Idag använder Parans en manuell metod för att kalibrera sensorn, flytta den punkt på koordinatsystemet som ljuset fokuserar ner till, genom att vrida solpanelen med hjälp av en terminalemulator och sedan kontrollera värdet på en separat luxmätare.
    % subsection bakgrund (end)

    \subsection{Syfte} % (fold)
    \label{sub:syfte}
          Syftet med projektet är att ta fram en helt automatisk process som kan kalibrera fotosensorn i Parans solpaneler, så att det maximala ljusflödet från panelen kan uppnås och detta med en lägre tidsåtgång och högre precision än dagens manuella metod. 
          Vidare syftar projektet till att möjliggöra kommunikation mellan panelen och en lux\-mätare inne i byggnaden, så att kalibreringstekniken kan nyttjas till systemen generellt, oavsett om de är i produktion eller i fabrik.
    % subsection syfte (end)

    \subsection{Mål} % (fold)
    \label{sub:mal}
        Målet med det här projektet är att ta fram ett automatiskt system som justerar fokuspunkten på ljussensorn, vilket då vrider på solpanelen för att lokalisera det X- och Y-värde där intaget av solljus är som störst. 
        Ljusstyrkan mäts med hjälp av en luxmätare som levererar ljusintaget till en dator eller till en annan programmerbar enhet. 
        När det maximala ljusintaget är uppmätt, registreras X- och Y-värdena som den nya fokuspunkten för ljussensorn, istället för det förinställda värdet på origo. 
        Vidare är målet att ta fram en from av kommunikation mellan en luxmätare inne i byggnaden och en panel som befinner sig på taket, så att även enheter som redan är satta i bruk kan kalibreras. 
    % section mal (end)


    \subsection{Frågeställning} % (fold)
    \label{sub:fragestallning}
        \begin{itemize}
            \item Vilken algoritm kan anses vara lämplig för kalibreringen?\\
            \item Vilka förutsättningar för kommunikation finns det mellan solpanelen och det upplysta rummet? \\
            \item Hur tillförlitligt är det valda kommunikationssättet? \\
            
        \end{itemize}
    % subsection fr_gest_llning (end)

    \subsection{Avgränsningar} % (fold)
    \label{sub:avgr_nsningar}
        \subsubsection{Hårdvara} % (fold)
        \label{ssub:h_rdvara}
            Redan existerande hårdvara kommer att användas, det vill säga sådan avsedd att användas för de ändamål nödvändiga för projektet. 
            Den primära hårdvaran, solpanel och luxmätare, kommer att tillhandahållas av uppdragsgivaren och inga alternativ till dessa kommer att undersökas. 
            Eventuell övrig hårdvara kan antingen vara helhetslösningar eller sådana som löser delproblem och kombineras. 
            Gällande lösningen som tas fram är den begränsad till att stödja företagets nu gällande panel SP3.\bigskip
        % subsection h_rdvara (end)

        \subsubsection{Mjukvara} % (fold)
        \label{ssub:mjukvara}
            Mjukvara kommer att utvecklas för att nå projektets uppsatta mål. 
            Denna kan komma att inkludera användning av båda medföljande och externa ramverk och bibliotek för att lösa olika delproblem, exempelvis grafisk framställning och kommunikation mellan olika enheter.
        % subsection mjukvara (end)

    % section avgr_nsningar (end)

% section indroduktion (end)