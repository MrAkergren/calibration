\section{Resultat} % (fold)
\label{sec:resultat}
    \subsection{Algoritm} % (fold)
    \label{sub:algoritm}

    % subsection algoritm (end)
    \subsection{Optisk kommunikation} % (fold)
    \label{sub:optisk_kommunikation}

        Kommunikationen mellan det upplysta rummet och solpanelen på taket kan upprättas med hjälp av två mikrokontroller, en för sändning av data och en för mottagande. \bigskip

        Den lösning som detta projekt presenterar består mikrokonrollerkortet av Arduino Uno revision 3 (för fullständig specifikation se bilaga \ref{sub:arduino_spec}).\cite{ardu} Till sändaren kopplas en lysdiod till det gränssnitt som skickar data via den seriella standarden, vilket då omvandlar RS-232 standardens höga- och låga läge till ljus på och ljus av. Till mottagaren kopplas en fotoresistor, en resistor som ändrar motståndet när den träffas av ljus, vilket gör att när den kopplas in till det seriella gränssnittet skapar resistorn spänningsförändringar som registreras som högt eller lågt värde av standarden.


    % subsection optisk_kommunikation (end)
    
% section resultat (end)