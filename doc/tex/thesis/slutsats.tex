\section{Slutsats} % (fold)
\label{sec:slutsats}
    \subsection{Sammanfattning} % (fold)
    \label{sub:sammanfattning}
        Detta projekt har kommit fram till att en automatiserad metod för att kalibrera Parans solpanel SP3 är möjlig och levererar en mjukvara som utför just denna uppgift. Krav för att kalibreringsmetoden ska fungera är att rätt hårdvara finns att tillgå, där kalibreringen i sig kräver en luxmätare för intag av belysningsgrad och en kommunikation till SP3an.\bigskip 

        För att kommunicera mellan panelen och det upplysta rummet har projektet lämnat två huvudsakliga förslag, dessa är dock inte lika klara till leverans så som algoritmen. Det behöver arbetas mer på den fysiska delen av kommunikationen, där fram för allt en fästanordning på panelen för att fästa en ljuskänslig mottagare behöver utvecklas. Även sändarna och mottagarna behöver designas, då dessa är i prototypstadiet med kopplingar på kopplingsbrädor. Teorin bakom kommunikationen är dock utredd och i laborationsmiljö har kommunikation upprättats via företagets eget kablage, så vidare utveckling av denna kommunikationsprincip är möjlig.
    % subsection sammanfattning (end)

    \subsection{Diskussion} % (fold)
    \label{sub:diskussion}
        Projektet kan anses ha två huvudsakliga syften, där det första är att ''ta fram en helt automatisk process som kan kalibrera fotosensorn i Parans solpaneler [~\dots~] med en lägre tidsåtgång och högre precision än dagens manuella metod''. 
        Den metod som bolaget tidigare använde sig av var dels baserad på manuell inmatning av värden, vilket tar tid och kan leda till fel på grund av den mänskliga faktorn, och dels på en manuell uppskattning av ljusstyrkan vilket också kan leda till en felaktig kalibrering. 
        Med hjälp av den algoritm som projektet har utvecklat och redovisat, i samverkan med den tekniska implementationen, anser författarna att detta syfte är uppnått. Processen kan skötas helt automatiskt så till vida att ljusflödet ut från panelen kan uppmätas. 
        Denna automatiserade kalibrering är att anse som tidsbesparande då inga värden behöver anges manuellt, särskilt då skillnaden mellan sensorns ursprungliga värde och det optimala inställningsvärdet och är stort så att många kalibreringssteg behöver göras. Den framtagna algoritmen är generell i det avseende att den inte är begränsad till användning för SP3-panelen utan bör även kunna användas för andra liknande produkter, såsom bolagets efterföljande modell SP4, eller tillämpningar där datastrukturer likt de illustrerade i figur~\ref{fig:array} kan påträffas. \bigskip

        Gällande bestämningen av ljusintensiteten finns det både för- och nackdelar med att göra en uppmätning av ljusstyrkan och en mänsklig uppskattning. Fördelarna med en automatiserad inläsning är att kalibreringen blir standardiserad och inte behöver bero på den person som utför kalibreringen. När författarna deltog i en manuell kalibrering av installerade paneler ute i produktion upplevde vi att ljusintensiteten varierar väldigt mycket, från bländande till i princip helt släckt och med tanke på att det mänskliga ögats anpassning till olika ljusintensiteter varierar beroende på om intensiteten ökar eller minskar kan kalibreringen tappa i precision vid en manuell bedömning \cite[s.~273]{aot}. Det är svårt att jämföra hur två inställningar förhåller sig till varandra, vilken som är starkare eller svagare, om ljuskällan blivit väldigt mörk mellan de båda tillfällena.\bigskip 

        Ett problem som kan uppstå är ifall projektets luxmätare är mer känslig för andra frekvenser än de frekvenser som det mänskliga ögat är känsligt för och om panelen vid felkalibrering tar in ett högre antal av de för luxmätarens känsliga frekvenser på grund av brytning vid linsen, så skulle mätaren registrera ett högre belysningsvärde än vad en människa skulle anse. Detta är en risk som projektet har accepterat då med stöd av litteratur som visar på att ljusmiljö är komplext att bedöma och kräver personer med erfarenhet för att bedömas korrekt, personer som företaget inte har att tillgå vid paneler redan satta i drift \cite[s.~278]{aot}. De kalibreringar som har genomförts i testmiljön har inte kunnat visa på att fel frekvenser skulle leda till en lägre belysningsgrad, dock finner vi att den uppmätta fokuspunkten, figur~\ref{fig:array1}, av panelen som märklig och luxmätarens frekvensspektra kan vara en orsak. Oavsett den mån teknikern skulle kunna bedöma belysningsgraden så finns vanligtvis inte möjlighet för denna att befinna sig i rummet dit ljuset leder då teknikern befinner sig vid panelen för att sköta kalibreringen. Detta leder då antingen till att teknikern behöver gå emellan panelen och det upplysta rummet, något som är väldigt tidsödande, eller att det krävs två personer för att utföra kalibreringen, en som sköter inmatningen till panelen och en som rapporterar ljusstyrkan. Sammantaget är vår bedömning att en uppmätning av ljusstyrkan är en lämplig metod då det sparar tid vid kalibreringen och resultatet blir oberoende av operatörers erfarenhet gällande bedömning av ljusintensitet. \bigskip

        Det bör dock påpekas att den ljusuppmätningen som detta projekt utför endast är till för att hitta det maximala ljuset ut från panelen och ska inte tas för en ljusmätning som kan representera belysningen i rummet. För att mäta upp belysningen av rummet behövs mer avancerade mätanordningar som kan ge korrekta värden.\bigskip

        Projektet syftar vidare till att ''möjliggöra kommunikation mellan panelen och en luxmätare inne i byggnaden'' där vi har undersökt två huvudsakliga metoder, trådlöst eller via de fiberoptiska kablarna som redan är dragna. Då företaget önskade att kommunicera med standardiserade trådlösa metoder för persondatorer uteslöts den möjligheten snart, i och med att hinder i form av betongkonstruktioner så som golv, väggar och tak befinner sig emellan panelen och luxmätaren, vilket hindrar spridningen av radiovågor och överföringen av data.\bigskip

        För optisk kommunikation lämnas dels ett förslag för att lösa detta delproblem och ett annat förslag som möjliggör att flytta luxmätaren från rummet upp till panelen, vilket då omintetgör behovet av att skicka datan från luxmätaren, eftersom den finns tillgänglig för den beräkningsenhet som utför algoritmen. Detta syfte är till del uppnått, då information kring hur dessa två förslag kan realiserats har redovisat, dock saknas det viss materiel för att kunna testa förslagen. Fotoresistorn som nyttjas vid mottagande av data behöver en infästning till panelen som är ljusskyddad samt eventuellt en lins för att fokuserar det utsända ljuset till resistorn. Denna infästning behöver tillverkas, men med avseende på projektets avgränsningar så anser vi att denna tillverkning ligger utanför projektets ramar. Är det möjligt att skicka data optiskt i det kablage som företaget använder i produktion? Ja det har bevisats möjligt, men har inte kunnat testas på deras produkt utan endast via kablaget.
    % subsection diskussion (end)

    \subsection{Arbetsgång} % (fold)
    \label{sub:arbetsg_ng}
        Den tidsplan som togs fram innan projektets start har i huvudsak följts och finns bifogad i bilaga~\ref{sec:tidsplan}. Vad vi i efterhand kan se är att inledande utveckling av algoritmen påbörjades projektvecka 2, veckan innan planerad utvecklingsstart, om än i väldigt skissartad form. Delvis berodde detta på ett behov av att acklimatisera sig till programmeringsspråket \texttt{Python}, av vilket tidigare erfarenhet var begränsad. Det kan också nämnas att fokus skiftade till rapportskrivning och utvärdering under projektvecka 8, något som i tidsplanen var planerat att ske projektvecka 9. Vid detta tillfälle hade kalibreringsapplikationen visat önskvärd funktion och projektets huvudmål hade nåtts. Då detta påskyndade dokumentationen gavs ytterligare tid till utvärdering och uppföljning.\bigskip

        \texttt{*** STYCKE OM ANGREPPSSÄTT OCH METOD HÄR ***}
    % subsection arbetsg_ng (end)

    \subsection{Vidareutveckling} % (fold)
    \label{sub:vidareutveckling}
        Det område som behöver vidareutvecklas från detta projekt är framförallt kommunikationsmöjligheten över optisk fiber. Författarna rekommenderar en undersökning i hur en fästanordning till solpanelen kan utformas på det lämpligaste sättet för att minska bakgrundsljus, samt hur mottagaren och sändaren kan utformas för att lämna prototypstadiet. Dessa förslag är att anses som utanför data- och informationstekniks institutionen utan är mer lämplig för ett designinriktat examensarbete. \bigskip

        Vidare är en undersökning av en modulerad sändning av ljus i det synliga spektrumet av intresse. Detta projekt har kännedom om grunderna men har inte undersökt det närmare på grund av avsaknad av materiel och tidsutrymme.
    % subsection vidareutveckling (end)
    
% section slutsats (end)