\section{Slutsats} % (fold)
\label{sec:slutsats}
    \subsection{Sammanfattning} % (fold)
    \label{sub:sammanfattning}
        Detta projekt har kommit fram till att en automatiserad metod för att kalibrera Parans solpanel SP3 är möjlig och levererar en mjukvara som utför just denna uppgift. Krav för att kalibreringsmetoden ska fungera är att rätt hårdvara finns att tillgå, där kalibreringen i sig kräver en luxmätare för avläsning och intag av belysningsgrad och en kommunikation till panelen SP3.\bigskip 

        För att kommunicera mellan panelen och det upplysta rummet har projektet lämnat två huvudsakliga förslag, dessa är dock inte lika långt utvecklade såsom algoritmen. Mer arbete behövs på den fysiska delen av kommunikationen där framför allt en fästanordning på panelen för att fästa en ljuskänslig mottagare behöver utvecklas. Även sändare och mottagare behöver designas, då dessa är i prototypstadiet med kopplingar på kopplingsbrädor. Teorin bakom kommunikationen är dock utredd och kommunikation har upprättats i laborationsmiljö via företagets eget kablage, så vidare utveckling av denna kommunikationsprincip är möjlig.
    % subsection sammanfattning (end)

    \subsection{Diskussion} % (fold)
    \label{sub:diskussion}

        Projektet kan anses ha två huvudsakliga syften, där det första är att ''möjliggöra en helt automatisk process som kan kalibrera fotosensorn i Parans solpaneler'' och att  ''minska tidsåtgången och höja  precisionen jämfört med dagens manuella metod''. 
        Den metod som bolaget tidigare använt sig av var dels baserad på manuell inmatning av värden, vilket tar tid och kan leda till fel på grund av den mänskliga faktorn, och dels på en manuell uppskattning av ljusstyrkan vilket också kan leda till en felaktig kalibrering. 
        Med hjälp av den algoritm som projektet har utvecklat och redovisat, i samverkan med den tekniska implementationen, anser författarna att detta syfte är uppnått. Processen kan skötas helt automatiskt under förutsättning att ljusflödet ut från panelen kan uppmätas. 
        Denna automatiserade kalibrering är att anse som tidsbesparande då inga värden behöver anges manuellt, särskilt då skillnaden mellan sensorns ursprungliga värde och det optimala inställningsvärdet och är stort så att många kalibreringssteg behöver göras. Den framtagna algoritmen är generell i det avseende att den inte är begränsad till användning för SP3-panelen utan bör även kunna användas för andra liknande produkter, såsom bolagets efterföljande modell SP4, eller tillämpningar där datastrukturer likt de illustrerade i figur~\ref{fig:array} kan påträffas.\bigskip

        \newpage
        Anledningen till att projektet inte fortsatte vidare i utvecklingen av algoritmen efter den tredje iterationen är förutsättningen att paneler satta tagna i bruk bör vara relativt kalibrerad från fabrik. Det vill säga att maxpunkten förväntas befinna sig nära nuvarande inställning, där en sökning av ett område större än 20$\times$20 söksteg ses som osannolik. Detta baserat både på de fokuspunkter som uppmätts i bilaga~\ref{sec:heatmap} och på uppgift från utvecklare inom företaget.  Denna grovkalibrering innebär att fortsatt förbättringsarbete inte skulle ge ett stort utslag på tidsbesparing, men skulle denna algoritm användas till helt okalibrerade sensorer kan vidare utveckling vara nödvändig. Exempel på vidare optimering kan vara större söksteg först, för att grovt söka igenom ljusbilden, innan en finare kalibrering vidtar. \bigskip

        Gällande resultat från körningar av algoritmen saknas utförlig sådan. Författarna är medvetna om att resultat gällande bland annat söktider hade varit lämpliga att redovisa i rapporten, men på grund av de väderförhållanden som rått under projekttiden har inga sådana resultat kunnat framställas. De enstaka dagar med ihållande solsken har ägnats åt att framställa de resultat som visas i kapitel~\ref{sec:resultat} och att utföra enstaka tester som visade på att algoritmen fungerar som tänkt.\bigskip

        Gällande bestämningen av ljusintensiteten finns det både för- och nackdelar med att göra en uppmätning av ljusstyrkan och en mänsklig uppskattning. Fördelarna med en automatiserad inläsning är att kalibreringen blir standardiserad och inte behöver bero på den person som utför kalibreringen. När författarna deltog i en manuell kalibrering av installerade paneler ute i produktion upplevde vi att ljusintensiteten varierar väldigt mycket, från bländande till i princip helt släckt och med tanke på att det mänskliga ögats anpassning till olika ljusintensiteter varierar beroende på om intensiteten ökar eller minskar kan kalibreringen tappa i precision vid en manuell bedömning \cite[s.~273]{aot}. Det är svårt att jämföra hur två inställningar förhåller sig till varandra, vilken som är starkare eller svagare, om ljuskällan blivit väldigt mörk mellan de båda tillfällena.\bigskip 

        Ett problem som kan uppstå är ifall projektets luxmätare är mer känslig för andra frekvenser än de frekvenser som det mänskliga ögat är känsligt för och om panelen vid felkalibrering tar in ett högre antal av de för luxmätarens känsliga frekvenser på grund av brytning vid linsen, så skulle mätaren registrera ett högre belysningsvärde än vad en människa skulle anse. Denna spektrumförskjutning kan vara en delförklaring till den avvikande fokuspunkten i figur~\ref{fig:ada}, där luxmätaren till synes ger större utslag för vissa frekvenser som ligger strax utanför fokuspunkten. Risken att luxmätarens utslag beror på aningen felaktiga ljusfrekvenser är något som projektet har accepterat, med stöd av litteratur. Litteraturen beskriver att ljusmiljö är komplext att bedöma och kräver personer med erfarenhet för att bedömas korrekt, personer som företaget inte har att tillgå vid paneler redan satta i drift \cite[s.~278]{aot}. De kalibreringar som har genomförts i testmiljön har inte kunnat visa på att fel frekvenser skulle leda till en lägre upplevd belysningsgrad. Oavsett den mån teknikern skulle kunna bedöma belysningsgraden så finns vanligtvis inte möjlighet för denna att befinna sig i rummet dit ljuset leder då teknikern befinner sig vid panelen för att sköta kalibreringen. Detta leder då antingen till att teknikern behöver gå emellan panelen och det upplysta rummet, något som är väldigt tidsödande, eller att det krävs två personer för att utföra kalibreringen, en som sköter inmatningen till panelen och en som rapporterar ljusstyrkan. Sammantaget är vår bedömning att en uppmätning av ljusstyrkan är en lämplig metod då det sparar tid vid kalibreringen och resultatet blir oberoende av operatörers erfarenhet gällande bedömning av ljusintensitet. \bigskip

        Det bör dock påpekas att den ljusuppmätning som detta projekt utför endast är till för att hitta det maximala ljuset ut från panelen och ska inte tas för en ljusmätning som kan representera belysningen i rummet. För att mäta upp belysningen av rummet behövs mer avancerade mätanordningar som kan ge korrekta värden.\bigskip

        Det andra huvudsakliga syftet till projektet är att ''möjliggöra kommunikation mellan panelen och en luxmätare inne i byggnaden'' där vi har undersökt två huvudsakliga metoder, trådlöst eller via de fiberoptiska kablarna som redan är dragna. Kommunikation med standardiserade trådlösa metoder för persondatorer uteslöts tidigt i arbetet, i och med att hinder i form av betongkonstruktioner såsom golv, väggar och tak befinner sig emellan panelen och luxmätaren, vilket hindrar spridningen av radiovågor och överföringen av data.\bigskip

        För optisk kommunikation lämnar författarna dels ett förslag för att lösa detta delproblem så som det är formulerat, genom att upprätta en datakommunikation mellan rummet och panelen, men även ett annat förslag lämnas som möjliggör att flytta luxmätaren från rummet upp till panelen. Det andra förslaget omintetgör då behovet av att skicka datan från luxmätaren, eftersom den finns tillgänglig för den beräkningsenhet som utför algoritmen.\bigskip

        Detta syfte är delvis uppnått då information kring hur dessa två förslag kan realiseras har redovisats, dock saknas viss materiel för att kunna testa förslagen. Fotoresistorn som nyttjas vid mottagande av data behöver en infästning till panelen som är ljusskyddad samt eventuellt en lins för att fokuserar det utsända ljuset till resistorn. Denna infästning behöver tillverkas, men med avseende på projektets avgränsningar så anser vi att denna tillverkning ligger utanför projektets ramar. Är det möjligt att skicka data optiskt i det kablage som företaget använder i produktion? Ja det har bevisats möjligt, men har inte kunnat testas på deras produkt utan endast via kablaget.\bigskip

        Ur ett hållbarhetsperspektiv har projektet arbetat med att förbättra en produkt, vars syfte är att höja ljuskvaliteten inomhus genom att leda in fullspektrumljus från solen till platser i byggnader där solen inte kan stråla in \cite{quality}. Ytterligare en aspekt av produkten är att energiförbrukningen för upplysning sänks \cite{panel_energy}, vilket leder till lägre kostnader och minskad klimatåverkan från de platser där panelen är installerad. 

    % subsection diskussion (end)

    \subsection{Arbetsgång} % (fold)
    \label{sub:arbetsg_ng}
        Den tidsplan som togs fram innan projektets start har i huvudsak följts och finns bifogad i bilaga~\ref{sec:tidsplan}. Vad vi i efterhand kunde se var att inledande utveckling av algoritmen påbörjades en vecka innan planerad utvecklingsstart, om än i väldigt skissartad form. Detta berodde dels på ett behov av att fördjupa sig i programmeringsspråket \texttt{Python}, av vilket tidigare erfarenhet var begränsad, och dels på att formulera tidiga tankar och idéer i kod. Det kan också nämnas att fokus skiftade till rapportskrivning och utvärdering en vecka tidigare än planerat. Vid detta tillfälle hade kalibreringsapplikationen visat önskvärd funktion och projektets huvudmål hade nåtts. Då detta påskyndade dokumentationen gavs ytterligare tid till utvärdering och uppföljning.\bigskip

        Författarna anser att den metod som projektet har nyttjat sig av har fungerat väl. Att fokusera på ett itererande sätt, där en artefakt kunde förbättras genom att utvärdera den efter varje förbättring, var gynnsamt och ledde till en naturlig utveckling av produkten. När det kommer till utvecklingen av kommunikationen och prioriteringen av vilken artefakt som borde ha utvecklats först kan beslutet kritiseras. Projektet valde att inrikta sig på seriell kommunikation för att undersöka om det fungerade via optisk överföring, när en frekvensmodulerad standard hade kunnat vara mer lämplig. Detta berodde delvis på idéer från företaget gällande vilken lösning de såg som intressant att testa men även på bristande förkunskap från projektet. Förslaget att testa frekvensmodulerad överföring kom under en pågående utvecklingsiteration av den seriella kommunikationen och när nästa utvecklingssteg skulle genomföras hade inköp och leverans av komponenter tagit för lång tid samt medfört en extra kostnad, vilket gjorde att den lösningen inte utvecklades. Istället för att försöka skicka data framfördes förslaget om att skapa en rundgång i systemet, vilket då nästa utvecklingssteg fokuserade på, på grund av lägre komplexitet och materialkostnad.
    % subsection arbetsg_ng (end)

    \subsection{Vidareutveckling} % (fold)
    \label{sub:vidareutveckling}
        Det område som framförallt behöver vidareutvecklas från detta projekt är kommunikationsmöjligheten över optisk fiber. Författarna rekommenderar en undersökning i hur en fästanordning till solpanelen kan utformas på lämpligast sätt för att minska bakgrundsljus, samt hur mottagaren och sändaren kan utformas för att lämna prototypstadiet. Dessa förslag är att anses som utanför institutionen för data- och informationsteknik och bör vara mer lämpligt för ett designinriktat examensarbete. \bigskip

        Vidare är en undersökning av modulerad sändning av ljus i det synliga spektret av intresse. Detta projekt har kännedom om grunderna men det har inte undersökts närmre på grund av avsaknad av materiel och tidsutrymme.
    % subsection vidareutveckling (end)
    
% section slutsats (end)