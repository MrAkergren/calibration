\renewcommand{\abstractname}{Abstract}
\begin{abstract}
    \noindent 
    % Purpose
    Parans Solar Lighting is a company that have developed a solar panel used to bring sunlight indoors, by focusing light into optical fibres. A photosensor is used to adjust the panel's direction, in order to maximize its intake of light, and the necessary calibration of the photosensor was previously performed manually.
    % Scope
    This thesis describes the development of an automatic procedure for the calibration of the photosensor, and focuses on two main problems. The first problem is an algorithm that performs the calibration with the aid of measured luminosity, and the second problem is a communication solution to transfer this information from the illuminated room to the panel.
    % Result
    The project resulted in a calibration software application and two suggestions of solutions regarding the communication, where the solar panels fibre cables are used as communication medium. \medskip

    \noindent \textbf{Keywords:} Calibration, RS-232, serial communication, optical commication, fibreoptics, microcontroller, indoor lighting, Parans, light sensor
\end{abstract}

\renewcommand{\abstractname}{Sammandrag}
\begin{abstract}
    \noindent
    % Purpose
    Företaget Parans Solar Lighting har utvecklat en solpanel som leder solljus genom optisk fiber för att användas till inomhusbelysning. Solpanelen använder en fotosensor för att maximera ljusintaget genom att justera dess riktning och en kalibrering av fotosensorn är nödvändig, vilket tidigare har utförts manuellt.
    % Scope
    Rapporten beskriver framtagandet av en automatisk kalibrering av fotosensorn och fokuserar på två huvudområden. Det första området är en kalibreringsalgoritm som använder värden av uppmätt ljusstyrka och det andra området är en lösning för kommunikation mellan solpanelen och det upplysta rummet, i syfte att förmedla denna information.
    % Result
    Resultaten av projektet var en mjukvarulösning som kalibrerar panelen och två förslag på lösningar gällande kommunikationen, där det rekommenderades att använda panelens egna fiberkablar som överföringsmedium. \medskip

    \noindent \textbf{Nyckelord:} Kalibrering, RS-232, seriell kommunikation, optisk kommunikation, fiberoptik, mikrokontroller, belysning, Parans, ljussensor
\end{abstract}

\newpage
\subsection*{Förord} % (fold)
\label{sub:f_rord}
    Detta examensarbete skrevs inom ramen för kursen LMTX38, Examensarbete vid Data- och informationsteknik vårterminen 2015 på Chalmers tekniska högskola. \bigskip

    Författarna vill rikta ett stort tack till Parans Solar Lighting AB för tillhandahållande av materiel, arbetsytor och stöd under projektets gång. Ett särskilt tack riktas till vår handledare från Parans, Karl Nilsson, för förslaget till examensarbetets problemområde och avsatt tid för handledning. Vi riktar även ett tack till Simon Larsson hos Parans för bollande av idéer och stöd under utvecklingsarbetet. Vidare vill vi tacka handledare Lennart Hansson vid Chalmers för hans stöd under denna rapports framställning och förslag om att driva arbetet framåt. Ett avslutande tack skickas till Sakib Sistek vid Chalmers för hans förmedling av kontakten till Parans och för hans hjälp vid förberedelsen till detta examensarbete.

% subsection f_rord (end)

\newpage

\subsection*{Beteckningar} % (fold)
\label{sub:beteckningar}
    \begin{tabularx}{\textwidth}{@{}rX}
        C & Programmeringsspråk \\
        I/O & Input/Output \\
        I²C & Standard för synkron seriell datakommunikation \\
        lm & lumen, SI-enhet för ljusflöde \\
        lux & SI-enhet för belysning, 1 lux = 1 $\sfrac{\text{lm}}{\text{m}^2}$ \\
        RS-232 & Standard för asynkron seriell datakommunikation \\
        UART & Universal Asynchronous Receiver/Transmitter, gränssnitt för seriell kommunikation
        
    \end{tabularx}
% subsection beteckningar (end)