\section{Diskussion} % (fold)
\label{sec:diskussion}

    Projektet är kan anses ha två huvudsakliga syften, där det första syftet är att ''ta fram en helt automatisk process som kan kalibrera fotosensorn i Parans solpaneler [~\dots~] med en lägre tidsåtgång och högre precision än dagens manuella metod''. 
    Den metod som företaget använde sig av tidigare, var dels baserad på manuell inmatning av värden, vilket tar tid och kan leda till fel på grund av den mänskliga faktorn och dels en manuell uppskattning av ljusstyrkan vilket även det kan leda till en felaktig kalibrering. 
    Med hjälp av den algoritm som projektet har utvecklat och redovisat, anser författarna att detta syfte är uppnått. Processen kan skötas helt automatiskt, så till vida att ljusflödet ut från panelen kan uppmätas. 
    Denna automatiserade kalibrering är att anses som tidsparande då inga värden behöver anges manuellt vilket sparar tid, särskilt då skillnaden mellan sensorns optimala inställningsvärde och det ursprungliga värdet är stort, så att många kalibrerings steg behöver göras. \bigskip

    Gällande bestämningen av ljusintensiteten finns det både för- och nackdelar mellan att göra en uppmätning av ljusstyrkan och en mänsklig uppskattning. Fördelarna med en automatiserad inläsning av ljusstyrkan är att kalibreringen blir standardiserad och inte behöver bero på personen som utför kalibreringen. När författarna deltog i ett praktiskt exempel av kalibrering av panelen ute i produktion upplevde vi att ljusintensiteten varierar väldigt mycket och med tanke på det mänskliga ögat och att dess anpassning till olika ljusintensiteter varierar olika snabbt beroende på om ljusintensiteten ökar eller minskar, kan kalibreringen tappa i precision vid en manuell bedömning.\cite[s.~273]{aot}  Det är svårt att jämföra hur två inställningar förhåller sig till varandra, vilken som är starkare eller svagare, om ljuskällan blivit väldigt mörk mellan de båda tillfällena. Det ska dock påpekas att en rent mekanisk bedömning har sina brister då ''[d]et är i det närmaste omöjligt att planera ljusmiljö [ \dots ] enbart med hjälp av fysikaliska mätningar'' men det påpekas också i litteraturen att det krävs erfarenhet för att kunna göra lämpliga bedömningar. \cite[s.~278]{aot} Personer med den erfarenheten finns sällan att tillgå för företaget då deras tekniker inte har möjlighet att befinna sig i rummet dit ljuset leder, utan befinner sig vid panelen för att sköta kalibreringen. Sammantaget är vår bedömning att en uppmätning av ljusstyrkan är den mest lämpliga metoden då det sparar tid vid kalibreringen och blir oberoende på operatörers erfarenhet gällande bedömning av ljusintensitet.

% section diskussion (end)