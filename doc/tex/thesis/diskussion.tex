\section{Diskussion} % (fold)
\label{sec:diskussion}

    Projektet kan anses ha två huvudsakliga syften, där det första är att ''ta fram en helt automatisk process som kan kalibrera fotosensorn i Parans solpaneler [~\dots~] med en lägre tidsåtgång och högre precision än dagens manuella metod''. 
    Den metod som bolaget använde sig av tidigare var dels baserad på manuell inmatning av värden, vilket tar tid och kan leda till fel på grund av den mänskliga faktorn, och dels på en manuell uppskattning av ljusstyrkan vilket också kan leda till en felaktig kalibrering. 
    Med hjälp av den algoritm som projektet har utvecklat och redovisat anser författarna att detta syfte är uppnått. Processen kan skötas helt automatiskt så till vida att ljusflödet ut från panelen kan uppmätas. 
    Denna automatiserade kalibrering är att anse som tidsbesparande då inga värden behöver anges manuellt, särskilt då skillnaden mellan sensorns ursprungliga värde och det optimala inställningsvärdet och är stort så att många kalibreringssteg behöver göras. \bigskip

    Gällande bestämningen av ljusintensiteten finns det både för- och nackdelar med att göra en uppmätning av ljusstyrkan och en mänsklig uppskattning. Fördelarna med en automatiserad inläsning är att kalibreringen blir standardiserad och inte behöver bero på den person som utför kalibreringen. När författarna deltog i en manuell kalibrering av installerade paneler ute i produktion upplevde vi att ljusintensiteten varierar väldigt mycket och med tanke på att det mänskliga ögats anpassning till olika ljusintensiteter varierar beroende på om ljusintensiteten ökar eller minskar kan kalibreringen tappa i precision vid en manuell bedömning \cite[s.~273]{aot}.  Det är svårt att jämföra hur två inställningar förhåller sig till varandra, vilken som är starkare eller svagare, om ljuskällan blivit väldigt mörk mellan de båda tillfällena. Det ska dock påpekas att en rent mekanisk bedömning har sina brister då ''[d]et är i det närmaste omöjligt att planera ljusmiljö [ \dots ] enbart med hjälp av fysikaliska mätningar'' men det påpekas också i litteraturen att det krävs erfarenhet för att kunna göra lämpliga bedömningar \cite[s.~278]{aot}. Personer med den erfarenheten finns sällan att tillgå för bolaget vid installationstillfället då installation i regel sker av lokalt anlitade tekniker. En analys av förväntad ljusmiljö är något som sker innan installation, ofta i diskussion med bolaget och arkitekter, så det bör vara utrett på förhand. Oavsett den mån installatören skulle kunna bedöma detta så finns vanligtvis inte möjlighet för denna att befinna sig i rummet dit ljuset leder då teknikern befinner sig vid panelen för att sköta kalibreringen. Detta leder då antingen till att teknikern behöver gå emellan panelen och det upplysta rummet, något som är väldigt tidsödande, eller att det krävs två personer för att utföra kalibreringen, en som sköter inmatningen till panelen och en som rapporterar ljusstyrkan. Sammantaget är vår bedömning att en uppmätning av ljusstyrkan är den mest lämpliga metoden då det sparar tid vid kalibreringen och resultatet blir oberoende av operatörers erfarenhet och uppskattning gällande bedömning av ljusintensitet.

% section diskussion (end)