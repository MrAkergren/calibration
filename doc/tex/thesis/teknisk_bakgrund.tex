\section{Teknisk bakgrund} % (fold)
\label{sec:teknisk_bakgrund}
    \subsection{Parans SP3} % (fold)
    \label{sub:parans_sp3}
        SP3 är tredje generationens solpanel utvecklad av Parans \cite{parans_manual}. Panelen monteras på utsidan av en byggnad, ofta på taket, och fokuserar solljus genom linser in i optisk fiber för att sedan genom armatur lysa upp inomhus. Varje panel har sex utgående kablar med fiberoptik, vardera ansluten till en armatur, vilkas räckvidd är upp till 20 meter. Två stegmotorer används för att justera panelens riktning horisontellt och vertikalt, styrda av ett mikrokontrollerkort, så att linserna alltid är vända mot solen. Motorernas rörelser bestäms av en algoritm i mjukvaran som räknar ut solens nuvarande position med hjälp av att kombinera mätvärden från en ljussensor och solens förväntade position på himlen baserat på tid, datum och installationsplatsens geografiska position angivet i longitud och latitud. Mjukvaran som körs på mikrokontrollern är skriven i \texttt{C}.

        \subsubsection{Mikrokontrollerkortet} % (fold)
        \label{ssub:mikrokontrollerkortet}
            Mikrokontrollerkortet som används i panelen är formgivet av Parans och är baserat på en PIC32 mikrokontroller. PIC32 är en kategori mikrokontroller tillverkade av Microchip Technology för användning i inbyggda system och ger tillgång till bland annat flera I/O-anslutningar och UART för seriell kommunikation \cite[s.~42--44]{PIC32}. För att kommunicera med mikrokontrollerkortet med en dator finns en USB-port som ger en seriell anslutning som hanteras av en UART-krets från Silicon Laboratories, CP2102. Detta kräver att den anslutna datorn har en drivrutin för CP2102 installerad och möjliggör anslutning via en terminalemulator för installation, diagnostik och underhåll.
        % subsubsection mikrokontrollerkortet (end)

        \subsubsection{Fiberoptik} % (fold)
        \label{ssub:fiberoptik}
            Den fiberoptiska kabel som används av Parans i dagsläget har en ljusöverföring om 96\% per meter och saluförs i fyra olika längder, 5, 10, 15 och 20 meter \cite{parans_spec}. Var kabel består av sex stycken fibrer och ger ett ljusflöde om 730 lm till 430 lm beroende på längd av kablaget.
        % subsubsection fiberoptik (end)

        \subsubsection{Ljussensor} % (fold)
        \label{ssub:ljussensor}
            För att optimera ljusintaget i panelens fibrer, finjusteras vinkeln till solen med hjälp av en ljussensor monterad på panelens front. Via en lins fokuseras ljuset ner till en punkt på sensorn, som skyddas av ett gråfilter för att dämpa värmeutvecklingen i fokuspunkten. Själva ljussensorn är tillverkad av Hamamatsu och sitter integrerad på en egendesignat mönsterkort. Det exakta databladet har ej kunnat lokaliserats för den använda sensorn, men den kan antas arbeta efter samma princip som sensorerna S5990-01 och S5991-01 \cite{hama}, enligt den information som finns inom Parans.
        % subsubsection ljussensor (end)
    % subsection parans_sp3 (end)
% section teknisk_bakgrund (end)