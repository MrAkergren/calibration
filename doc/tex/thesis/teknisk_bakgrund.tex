\section{Teknisk bakgrund} % (fold)
\label{sec:teknisk_bakgrund}
    \subsection{Parans SP3} % (fold)
    \label{sub:parans_sp3}
        SP3 är tredje generationens solpanel utvecklad av Parans \cite{parans_manual}. Panelen monteras på utsidan av en byggnad, ofta på taket, och fokuserar solljus genom linser in i optisk fiber för att sedan genom armatur lysa upp inomhus. Varje panel har sex utgående kablar med fiberoptik, vardera ansluten till en armatur, vilkas räckvidd är upp till 20 meter. Panelen är utformad så att ultraviolett och infrarött ljus avskärmas från det ljus som leds in i fibern. Vid fabriksmontering limmas fibrerna fast i panelen och dessa kan sedermera ej enkelt avlägsnas. Diagnostik av ljusintaget är därför inte möjligt vid panelen, utan kan endast ske vid fiberkablarnas ändar.\bigskip

        Två stegmotorer används för att justera panelens riktning horisontellt och vertikalt så att linserna alltid är vända mot solen. Motorernas rörelser styrs av ett mikrokontrollerkort där en algoritm i mjukvaran räknar ut solens nuvarande position. Algoritmen kombinerar mätvärden från en ljussensor med solens förväntade himlaposition, baserat på tid, datum och installationsplatsens geografiska position. Mjukvaran som körs på mikrokontrollern är skriven i \texttt{C}.

        \subsubsection{SP3 mikrokontrollerkort} % (fold)
        \label{ssub:sp3_mikrokontrollerkort}
            Mikrokontrollerkortet som används i panelen är konstruerat av Parans och är baserat på en PIC32-mikrokontroller. PIC32 är en kategori mikrokontroller tillverkade av Microchip Technology för användning i inbyggda system och ger tillgång till bland annat I/O-anslutningar och UART för seriell kommunikation \cite{PIC32}. För att kommunicera med mikrokontrollerkortet med en dator finns en USB-port som ger en seriell anslutning som hanteras av en UART-krets från Silicon Laboratories, CP2102. Detta kräver att den anslutna datorn har en drivrutin för CP2102 installerad och möjliggör anslutning via en terminalemulator för installation, diagnostik och underhåll.
        % subsubsection sp3_mikrokontrollerkort (end)

        \subsubsection{Fiberoptik} % (fold)
        \label{ssub:fiberoptik}
            Den fiberoptiska kabel som i dagsläget används av Parans är en plastfiber som har en ljusöverföring om 96 \% per meter. Panelen saluförs med fiberkablar i fyra fasta längder, 5, 10, 15 och 20 meter \cite{parans_spec}. Var kabel består av sex stycken fibrer och ger ett ljusflöde om 730~lm till 430~lm från solpanelen i fullt solljus, beroende på längd av kablaget. Vid fibrernas paneländar finns IR-speglar monterade som reflekterar den infraröda delen av solljuset och de används för att fibern annars riskerar att smälta.
        % subsubsection fiberoptik (end)

        \newpage
        \subsubsection{Ljussensor} % (fold)
        \label{ssub:ljussensor}
            För att optimera ljusintaget i panelens fibrer finjusteras vinkeln till solen med hjälp av en ljuskänslig krets monterad på panelens front. Kretsen skyddas av ett gråfilter med 10 \% ljusgenomsläpp för att dämpa solljusets intensitet. Via en lins fokuseras genomträngande ljus till en fokuspunkt på sensorn som ger ett utslag och kretsen omvandlar denna data till fyra strömmar, vilkas värden representerar avståndet från sensorns mittpunkt. Strömmarna omvandlas och representeras som x- och y-värden i ett koordinatsystem, värdena kan sedan avläsas och manuellt justeras vid kommunikation med panelen. Själva ljussensorn är tillverkad av Hamamatsu, med beteckningen S5901, och sitter integrerad på ett egendesignat mönsterkort. Databladet för den använda sensorn har ej kunnat lokaliseras men enligt information inom Parans kan den antas arbeta efter samma princip som sensorerna S5990-01 och S5991-01 \cite{hama}.
        % subsubsection ljussensor (end)
    % subsection parans_sp3 (end)

    \subsection{Luxmätare} % (fold)
    \label{sub:luxm_tare}
    För att mäta upp ljusstyrkan från panelen används en luxmätare kopplat till en av de fiberkablar som leder ner till det upplysta rummet. De luxmätare som stöds av projektet är mätare avsedda för privatbruk och är inte att anse som professionella i det avseende att ljusmiljön i ett rum kan bestämmas med hjälp av dem. Syftet med mätarna är istället att registrera skillnaden i ljusstyrkan från panelen. När panelen är rätt kalibrerad kommer luxmätaren leverera ett högre värde än vid en felkalibrerad panel. Det exakta värdet är i detta fall inte av intresse, det är istället möjligheten att hitta den inställning på panelen som ger luxmätarens maximala värdet.

        \subsubsection{Adafruit TSL2591 Digital Light Sensor}
            \label{ssub:ada_tsl2591}
            Sensorkortet från Adafruit innehåller ljussensorn TSL2591 från tillverkaren ams och används till att uppmäta ljusintensitet upp till 88\thinspace000 lux och kan anslutas till en mikrokontroller via I²C. På sensorn finns två fotodioder, där den ena reagerar på IR-ljus och den andra reagerar på fullspektrumljus \cite{TSL2591}. Fotodiodernas avläsning kan ske oberoende av varandra eller kombineras vilket möjliggör avläsning av endast synligt ljus. Sensorns värden avläses digitalt.
            % subsubsection ada_tsl2591 (end)
        % subsection arduino (end)

        \subsubsection{Yoctopuce Yocto-Light-V3} % (fold)
        \label{sub:yocto}
            Yocto-Light-V3 är en luxmätare i form av ett kretskort, baserad på ljussensorn BH1751FVI från ROHM, som är avsedd att mäta synligt ljus upp till 100\thinspace000 lux \cite{yocto}. Kretskortet har en USB-port för anslutning till dator och kräver ingen extra drivrutin mer än de som medföljer vanliga operativsystem för att användas. Tillverkaren Yoctopuce tillhandahåller kodbibliotek till flera vanligt förekommande programmeringsspråk som möjliggör avläsning av sensorvärden.
        % subsection yocto (end)
    % subsection luxm_tare (end)
    \newpage
    \subsection{Arduino} % (fold)
    \label{ssub:arduino_uno}
        Arduino Uno är ett mikrokontrollerkort byggt kring mikrokontrollern ATMega328. På kortet finns bland annat USB-anslutning och 20 pins för att ansluta externa enheter och kringutrustning, 6 analoga och 14 digitala. \cite{ard_internals}. Mikrokontrollern är en del av AVR-serien från Atmel och den har en RISC-baserad processor, 32 KB flashminne för lagring av programkod och 2 KB internminne. ATMega328 har också en UART som möjliggör seriell kommunikation. Plattformen för Arduino, där Uno är en implementering, är öppen och det är fritt att bygga mikrokontrollerkort enligt tillgängliga scheman. Till Arduino tillhandahålls även en tillhörande utvecklingsmiljö som möjliggör programmering via en dator och ger tillgång till exempelkod och färdigskrivna kodbibliotek. Kod till Arduinoenheter skrivs i Arduinos egen implementation av språken \texttt{C} och \texttt{C++} \cite{ard_c, ard_cplusplus}. För fullständig specifikation, se bilaga~\ref{sub:arduino_spec}.
    % subsection arduino_uno (end)
% section teknisk_bakgrund (end)